\svnid{$Id$}

\Modultitel{Digitale Medien}

\Modulkuerzel{72024}

\EnglischerTitel{Digital Media}

\SWS{3}

\ECTS{4}

\Sprache{Deutsch}

\Moduldauer{1}

\Turnus{
\periodisch{\SoSe}
}

\Modulverantwortlicher{\StudiendekanInf}

\Dozenten{
\Prof{Dr. Michael Weber}
}

\Einordnung{
\MedInf{\Ba}{\Pflichtfach}{\MEI}
}

\VoraussetzungenInhaltlich{Keine}
\VoraussetzungenFormal{Keine}

\Lernziele{
Die Teilnehmer beherrschen die grundlegenden Konzepte der Verarbeitung, Speicherung,
Präsentation und Kommunikation medialer Daten in computerbasierten Systemen. Sie sind in der Lage
die wichtigsten Algorithmen zu den verschiedenen Medientypen zu nutzen und selbst zu implementieren. 
Sie sind in der Lage die jeweiligen Vor- und Nachteile bestimmter Verfahren zu analysieren und zu bewerten. 
}

\Inhalt{
\spiegelstrich{Grundlagen: Digitale Repräsentation, Kapazitätsanforderungen digitaler Medien, Kodierung und Kompression}
\spiegelstrich{Text: Schrift und Alphabet, Kodierungsstandards, Schriftart, Fortgeschrittene Textrepräsentationen}
\spiegelstrich{Grafik: 2D Formen und Operationen / Animation, 3D Formen und Operationen, Realismus}
\spiegelstrich{Bild: Bildkompression, Bildverarbeitung}
\spiegelstrich{Video: Analoges Video, Digitales Video, Videokompression}
\spiegelstrich{Audio: Physikalische und physiologische Grundlagen, Audiokodierung und Kompression, Repräsentation von Musik}
}

\Literatur{
\buch {Nigel Chapman and Jenny Chapman: Digital Multimedia. 2nd Edition, John Wiley \& Sons, 2004}
\buch {James D Foley Andries van Dam, Steven K Feiner John F Hughes: D. Foley, Dam K. Feiner, F. Computer Graphics. 2nd Edition, Addison-Wesley, 1996}
\buch {Peter A Henning: Taschenbuch Multimedia. A. Multimedia Hanser Fachbuchverlag, 2003}
\buch {Ze-Nian Li, Mark S. Drew: Fundamentals of Multimedia. Pearson, Pearson Prentice Hall 2004 Hall, 2004}
\buch {Ralf Steinmetz: Multibook. Springer-Verlag, 2000}
}

\Grundlagen{--}

\Lehrformen{
\Vlg{Digitale Medien}{Prof. Dr. Michael Weber}
\Ubg{Digitale Medien}{Dipl.-Ing. Frank Honold}
\Ubg{Digitale Medien}{Dipl.-Ing. Felix Schüssel}
}

\Arbeitsaufwand{
\Praesenzzeit{45}    
\VorNachbereitung{75}
\Summe{120}
}

\Leistungsnachweis{Die Modulprüfung erfolgt schriftlich.}

\Notenbildung{Die Modulnote ergibt sich aus der Modulprüfung.
Bei erfolgreicher Teilnahme an den Übungen wird dem Studierenden ein Notenbonus gemäß \S13 (5) der fachspezifischen Prüfungsordnung Informatik/Medieninformatik/Software Engineering gewährt.
}

%\Grundlagen{}


