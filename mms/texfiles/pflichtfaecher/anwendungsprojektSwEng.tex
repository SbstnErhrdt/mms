\svnid{$Id$}

\Modultitel{Anwendungsprojekt Software-Engineering}

\EnglischerTitel{Software Engineering Project}

\Modulkuerzel{71602}

\Sprache{Deutsch}

\SWS{6}

\ECTS{12}

\Moduldauer{2}

\Turnus{
  \periodisch{Semester}
}

\Modulverantwortlicher{
\StudiendekanInf
}

\Dozenten{
  \Prof{Dr. Helmuth Partsch}
  \LB{Dr. Alexander Raschke}
}

\Einordnung{
  \SwEng{\Ba}{\Anwendungsprojekt}{Software Engineering}
}

\VoraussetzungenInhaltlich{
  Modul Softwaretechnik, Modul Softwaregrundprojekt
}

\VoraussetzungenFormal{
  keine
}

\Lernziele{
  Die Studierenden haben einschlägige Kenntnisse über Softwareentwicklung und kennen insbesondere die wesentlichen Techniken (Formalismen, Vorgehensweisen und Werkzeuge) die dabei eingesetzt werden. Sie sind in der Lage sich in eine neue Aufgabenstellung einzuarbeiten und dafür geeignete Techniken auszuwählen. Sie können diese im Team einsetzen und verstehen es innovative Lösungen nach den Prinzipien des Software Engineering zu entwickeln. Sie haben einschlägige Erfahrungen im Umgang mit Software-Entwicklungswerkzeugen sowie mit Teamarbeit, Präsentationstechniken, schriftlicher Dokumentation und Techniken der Projektabwicklung.
}

\Inhalt{
Im Projekt wird ein nichttriviales Software-System aus einem speziellen Anwendungsgebiet nach den Prinzipien des Software Engineering im Team entwickelt. Dabei sind die Studierenden in Absprache mit ihrem Betreuer
für alle Aspekte der Projektabwicklung verantwortlich, und zwar von der Projektplanung über die Anforderungsdefinition und die eigentliche Entwicklung (nach einem geeigneten Vorgehensmodell) bis zu einem wohl-strukturierten, sauber dokumentierten, lauffähigen System (einschließlich ausgiebiger Qualitätskontrollen und Dokumentation).
}

\Literatur{
  \aufsatz{Wissenschaftliche Aufsätze aus einschlägigen Zeitschriften und Konferenzen}
  \aufsatz{Technische Dokumentation}
}

\Lehrformen{
  \Prj{Anwendungsprojekt 1}{Dr. Alexander Raschke}
  \Prj{Anwendungsprojekt 2}{Dr. Alexander Raschke}
}

\Arbeitsaufwand{
  \Praesenzzeit{50}    
  \VorNachbereitung{250}
  \Summe{360}
%   \Aufwand{Selbststudium, Literaturstudium, Besprechung mit dem Betreuer}{50}
%   \Aufwand{Durchführung der praktischen Tätigkeit}{250}
%   \Aufwand{schriftliche Ausarbeitung und Abschlusspräsentation}{60}
%   \Summe{360}
}



\Leistungsnachweis{
  Die Vergabe der Leistungspunkte erfolgt für die erfolgreiche Abnahme der Projektdokumentation und des entwickelten Systems.
}

\Notenbildung{
  Das Modul ist unbenotet.
}

\Grundlagen{
  Bachelorarbeit in Software Engineering
}

