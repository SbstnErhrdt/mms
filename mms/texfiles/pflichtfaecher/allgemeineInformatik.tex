\svnid{$Id$}

\Modultitel{Allgemeine Informatik (I,II)}

\Modulkuerzel{70002}

\EnglischerTitel{}

\SWS{6}

\ECTS{12} 

\Sprache{Deutsch}

\Modulverantwortlicher{\StudiendekanInf}

\Moduldauer{2}

\Turnus{
\periodisch{\WiSe}
}

\Dozenten{
\LB{Axel Fürstenberger}
\LB{Dr. Klaus Murmann}
}
\Einordnung{
\ET{\Ba}{\Pflichtmodul}{Elektrotechnik}
}

\VoraussetzungenInhaltlich{keine}
\VoraussetzungenFormal{Keine}

\Lernziele{Studierende, die dieses Modul erfolgreich absolviert haben,
\spiegelstrich{kennen Grundlagen formaler Sprachen und ihre Definition.}
\spiegelstrich{können mit Rechnern, Betriebssystemen, Dienstprogrammen und Werkzeugen praktisch umgehen.}
\spiegelstrich{besitzen Einsicht und Intuition in der Konstruktion von Algorithmen anhand konkreter Beispiele.}
\spiegelstrich{können Algorithmen anhand von Komplexitätsuntersuchungen beurteilen.}
}
\InhaltForts{
\spiegelstrich{sind in der Lage, in einer modernen Programmiersprache einfache Algorithmen systematisch zu entwickeln und in ein lauffähiges Programm umzusetzen, kennen und verstehen komplexere Datenstrukturen wie etwa Bäume oder assoziative Arrays in Definition (Rekursion) und Anwendung (rekursive Algorithmen).}
\spiegelstrich{können die Prinzipien moderner Modellierungstechniken verstehen und auf der Ebene einfacher Aspekte anwenden.}
\spiegelstrich{kennen klassische wie auch moderne Programmierparadigmen (z.B. Rekursion, Abstrakte Datentypen, Vererbung, Polymorphie, Ausnahmenbehandlung) und können diese auch praktisch anwenden.}
}

\Inhalt{
In diesem Modul werden folgende fachliche Inhalte vermittelt:
\spiegelstrich{Einführung in das verwendete Betriebssystem, Behandlung nützlicher Kommandos und Dienstprogramme sowie praktischer Umgang mit Dateien und Prozessen}
\spiegelstrich{Formale Sprachen: Definition und Strukturierung}
\spiegelstrich{Reguläre Ausdrücke, endliche Automaten}
\spiegelstrich{Algorithmen und Komplexität}
\spiegelstrich{Prinzipien der Systementwicklung und -strukturierung}
\spiegelstrich{Typen von Programmiersprachen}
\spiegelstrich{Standarddatentypen, einfache strukturierte Datentypen sowie Kontrollstrukturen der gewählten Programmiersprache}
\spiegelstrich{Entwicklung von einfachen Algorithmen für Standardprobleme (z.B. Suchen, Sortieren)}
\spiegelstrich{Strukturierung von Software im Großen}
\spiegelstrich{Komplexe Datenstrukturen (z.B. Listen, Bäume) und Algorithmen darauf}
\spiegelstrich{Moderne Programmiersprachenkonzepte wie Vererbung oder Polymorphie}
\spiegelstrich{Aspekte der Verlässlichkeit (z.B. Ausnahmenbehandlung)}
}

\Literatur{
\buch{Knuth, D.: The Art of Computer Programming, Fundamental Algorithms; Addison-Wesley}
\buch{Wirth, N.: Algorithmen und Datenstrukturen; Teubner Verlag}
\buch{Lang, H.W.: Algorithmen und Datenstrukturen in Java; Oldenbourg}
\buch{Sedgewick, R.: Algorithmen in Java; Pearson Studium 2003}
}

\Grundlagen{--}

\Lehrformen{
\Vlg{Allgemeine Informatik I, 2 SWS}{}
\Ubg{Allgemeine Informatik I, 1 SWS}{}
\Vlg{Allgemeine Informatik II, 2 SWS}{}
\Ubg{Allgemeine Informatik II, 1 SWS}{}
}

\Arbeitsaufwand{
\Praesenzzeit{90}    
\VorNachbereitung{270}
\Summe{360}
}

\Leistungsnachweis{Die Vergabe der Leistungspunkte erfolgt aufgrund des Bestehens je einer
schriftlichen Modulteilprüfung in den beiden Lehrveranstaltungen Allgemeine
Informatik I und II. Die Anmeldung zu jeder dieser Modulteilprüfungen setzt
einen Leistungsnachweis voraus (Erreichen von 50 % der Punkte in den
Übungsaufgaben).}

\Notenbildung{Die Modulnote ergibt sich als leistungspunktgewichtetes Mittel aus den
Ergebnissen der Modulteilprüfungen.
}
