\svnid{$Id$}

\Modultitel{Logik, Berechenbarkeit und Komplexit\"at}

\Modulkuerzel{72016}

\EnglischerTitel{Logic, Computability and Complexity}

\SWS{6}

\ECTS{8}

\Sprache{Deutsch}

\Modulverantwortlicher{\StudiendekanInf}

\Moduldauer{1}

\Turnus{
\periodisch{\SoSe}
}

\Dozenten{
\Prof{Dr. Uwe Sch\"oning}
\Prof{Dr. Jacobo Tor\'an}
\Prof{Dr. Enno Ohlebusch}
}
\Einordnung{
\Inf{\Ba}{\Pflichtfach}{\TMI}
\MedInf{\Ba}{\Schwerpunkt}{}
\SwEng{\Ba}{\Pflichtfach}{\SoftwareEngineering}
}

\VoraussetzungenInhaltlich{Modul Formale Grundlagen, Modul Algorithmen und Datenstrukturen}
\VoraussetzungenFormal{Keine}

\Lernziele{Die Studierenden erwerben fundierte Kenntnisse zu den Grundlagen 
und der praktischen Relevanz der Logik unter besonderer Ber\"ucksichtigung der 
Informatik. Sie verstehen und erkl\"aren logisches Schlie\ss en.
Die Studierenden k\"onnen die vorgestellten Logikkalk\"ule kritisch reflektieren,
insbesondere hinsichtlich Komplexit\"at, Korrektheit und 
Vollst\"andigkeit. Sie sind in der Lage, Problemspezifikationen in
Logikprogramme umzusetzen und beherrschen praktische Aspekte der
Programmierung in Prolog. Die Studierenden erwerben fundierte Kenntnisse \"uber die  Formalisierung des Algorithmenkonzepts und des Effizienzbegriffs.
Sie sind  in der Lage, algorithmische  Aufgabestellungen gem\"a\ss\ ihrer effizienten L\"osbarkeit  einzuordnen und besitzen anderseits Kenntnisse \"uber die 
grunds\"atzlichen Grenzen der  Algorithmisierbarkeit.
}

\Inhalt{
Im Modul werden Begriffe, Methoden und Resultate der formalen 
Logik vorgestellt, die in verschiedenen Gebieten der Informatik Anwendung 
finden.
\spiegelstrich{Aussagenlogik: Syntax und Semantik, Normalformen,
Erf\"ullbarkeitsproblem, Hornformeln und Markierungsalgorithmus, Resolution, 
Endlichkeitssatz.}
\spiegelstrich{Pr\"adikatenlogik erster Stufe: Syntax und Semantik,
Normalformen, Skolemform, Erf\"ullbarkeitsproblem, Formalisierung des 
Folgerungsbegriffs, Unifikation, Korrektheit und Vollst\"andigkeit des 
Resolutionskalk\"uls.}
\spiegelstrich{Logische Programmierung: Theoretische Grundlagen und praktische
Programmierung in Prolog.}
Im Modul werden die grundlegenden Begriffe aus den  Gebieten der Berechenbarkeits- und Komplexitätstheorie eingef\"uhrt.
\spiegelstrich{Berechenbarkeit: Intuitiver Berechenbarkeitsbegriff und  Churchsche These,
verschiedene \"aquivalente Formalisierungen der Berechenbarkeitskonzepts,  
Halteproblem, Reduzierbarkeit und Unentscheidbarkeit.}
\spiegelstrich{Komplexit\"atstheorie: Komplexit\"atsklassen, das P-NP Problem,
NP-Vollst\"andigkeit, Komplexit\"at von Graphenproblemen.}
}

\Literatur{
\skript{Vorlesungsskript Logik}
\buch{U. Sch\"oning: Logik f\"ur Informatiker. 5. Auflage, Spektrum Verlag, 2000}
\buch{J. Dassow: Logik f\"ur Informatiker. Teubner, 2005}
\buch{B. Heinemann, K. Weihrauch: Logik f\"ur Informatiker. Teubner, 1992.}
\skript{Vorlesungsskript berechenbarkeit und Komplexität}
\buch{U. Sch\"oning: Theoretische Informatik -- kurz gefasst, Spektrum  Verlag, 5. Auflage, 2008.}
\buch{M. Sipser: Introduction to the Theory of Computation; PWS Publ.  Company, 1997.}
}

\Grundlagen{--}

\Lehrformen{
\Vlg{Logik}{}
\Ubg{Logik}{}
\Vlg{Berechenbarkeit und Komplexit\"at}{}
\Ubg{Berechenbarkeit und Komplexit\"at}{}
}

\Arbeitsaufwand{
\Praesenzzeit{90}
\VorNachbereitung{150}
\Summe{240}
}

\Leistungsnachweis{Die Modulpr\"ufung erfolgt schriftlich.}

\Notenbildung{Die Modulnote ergibt sich aus der Modulprüfung.
Bei erfolgreicher Teilnahme an den Übungen wird dem Studierenden ein Notenbonus gemäß \S13 (5) der fachspezifischen Prüfungsordnung Informatik/Medieninformatik/Software Engineering gewährt.
}

