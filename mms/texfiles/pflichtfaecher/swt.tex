\svnid{$Id$}

\Modultitel{Softwaretechnik}

\EnglischerTitel{Software Engineering}

\Modulkuerzel{71592}

\Sprache{deutsch}

\SWS{4}

\ECTS{6}

\Moduldauer{2}

\Turnus{
  \periodisch{\WiSe}
}

\Modulverantwortlicher{
\StudiendekanInf
}

\Dozenten{
  \Prof{Dr. Helmuth Partsch}
}

\Einordnung{
  \Inf{\Ba}{\Pflichtfach}{\PAI}
  \MedInf{\Ba}{\Pflichtfach}{\PAI}
  \SwEng{\Ba}{\Pflichtfach}{\PAI}
  \IST{\Ba}{\Wahlpflichtmodul}{}
  \Inf{\La}{\Pflichtmodul}{}
}

\VoraussetzungenInhaltlich{
 Modul Programmieren von Systemen
}

\VoraussetzungenFormal{
  keine
}

\Lernziele{
  Die Studierenden haben ein Bewusstsein für die Bedeutung, Schwierigkeiten und Möglichkeiten des Software Engineering sowie einschlägige Kenntnisse über Software, Softwareentwicklung, Softwarequalität und Projektmanagement. Sie wissen, dass erfolgreiches Software Engineering sorgfältige Planung, systematische Vorgehensweise und Disziplin erfordert. Sie wissen außerdem, dass gründliches und systematisches Requirements Engineering sowie sorgfältiger Grob- und Feinentwurf unabdingbar für den Erfolg eines Softwareprojekts sind und kennen entsprechende Techniken. Sie kennen auch die wichtigsten Qualitätssicherungsmaßnahmen, sind in der Lage, gängige Qualitätssicherungsmaßnahmen sinnvoll einzuplanen und können diese umsetzen. Sie kennen außerdem die wesentlichen Aspekte des Projektmanagements und Techniken zur Lösung der dabei anfallenden Aufgaben. Sie wissen, welche nicht-fachlichen Schwierigkeiten (z.B. Zeitökonomie, Kommunikations- und Abstimmungsprobleme, Schwierigkeiten in der Zusammenarbeit mit anderen) im Rahmen der Software-Erstellung auftreten können und wie man erfolgreich damit umgeht.
}

\Inhalt{
  Die Vorlesung gibt einen Überblick über alle relevanten Themen des Software Engineering. Insbesondere werden behandelt:
  \spiegelstrich {Motivation und Einführung in die Problemstellung}
  \spiegelstrich {Systems-Engineering, Vorgehensmodelle}
  \spiegelstrich {Softwareerstellung (Requirements Engineering, Entwurf, Implementierung, Werkzeuge)}
  \spiegelstrich {Qualitätssicherung (Metriken, Systematisches Testen, Reviews)}
  \spiegelstrich {Projektmanagement (Planung, Kostenschätzung, Controlling, Konfigurationsmanagement, Qualitätsmanagement, Prozessverbesserung)}
}

\Literatur{
  \skript{Kopien der Vorlesungsfolien}
}

\Lehrformen{
  \Vlg{Softwaretechnik 1, 2 SWS}{Prof. Dr. Helmuth Partsch}
  \Vlg{Softwaretechnik 2, 2 SWS}{Prof. Dr. Helmuth Partsch}
}

\Arbeitsaufwand{
  \Praesenzzeit{60}
  \VorNachbereitung{120}
  \Summe{180}
}

\Leistungsnachweis{
  Die Modulprüfung erfolgt schriftlich.
}

\Notenbildung{
  Die Modulnote ergibt sich aus der Modulprüfung.
}

\Grundlagen{
  Modul Anwendungsprojekt Software-Engineering
}


