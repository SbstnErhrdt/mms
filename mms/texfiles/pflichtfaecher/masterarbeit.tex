\svnid{$Id$}

\Modultitel{Masterarbeit}

\EnglischerTitel{Master's Thesis}

\Modulkuerzel{80000}

\Sprache{deutsch}

\SWS{0}

\ECTS{30}

\Moduldauer{1}

\Turnus{
  \periodisch{\Se}
}

\Modulverantwortlicher{
  \StudiendekanInf
}

\Dozenten{
  \alleDoz{Erstbetreuer der Masterarbeit}
}

\Einordnung{
  \Inf{\Ma}{\Abschlussarbeit}{Masterarbeit}
  \MedInf{\Ma}{\Abschlussarbeit}{Masterarbeit}
}

\VoraussetzungenInhaltlich{
Mindestens die laut Prüfungsordnung zu belegenden Module der Kernfächer. Wünschenswert ist es, im Vertiefungsfach grundlegende Module aus dem geplanten Gebiet der Masterarbeit belegt zu haben.
}

\VoraussetzungenFormal{
  keine
}

\Lernziele{
Selbstständiges Einarbeiten und wissenschaftlich methodische Bearbeitung eines für die Informatik bzw. Medieninformatik relevanten Themas. Erwerb der Fähigkeiten, komplexe Fragestellungen der Informatik bzw. Medieninformatik unter Anwendung des erlernten Fachwissens sowie bekannter wissenschaftlicher Methoden und Erkenntnisse innerhalb eines vorgegebenen Zeitrahmens selbständig zu bearbeiten, in Form einer Ausarbeitung darzustellen und vor sachkundigem Publikum verständlich zu präsentieren. Erlernen von Schlüsselqualifikationen wie Management eines eigenen Projekts, Präsentationstechnik und Verfeierung der rhetorischen Fähigkeiten.
}

\Inhalt{
  Abhängig von der konkreten Themenstellung.
}

\Literatur{
  Abhängig von der konkreten Themenstellung.
}

\Grundlagen{--}

\Lehrformen{
  \MaArb{Wahl eines geeigneten Themas an einem der Institute der
  Informatik}{Dozenten der Informatik} 
}

\Arbeitsaufwand{
  \Praesenzzeit{10}
  \VorNachbereitung{890}
  \Summe{900}
}

\Leistungsnachweis{
  Schriftliche Ausarbeitung und Abschlussvortrag.
}

\Notenbildung{
  Die Modulnote wird gemäß Prüfungsordnung gebildet.
}


