\svnid{$Id$}

\Modultitel{Grundlagen der Rechnernetze für Ingenieure}

\Modulkuerzel{????? (Wird vom Dezernat 2 festgelegt)}

\EnglischerTitel{Introduction to Computer Networks for Engineers}

\SWS{3}

\ECTS{4} 

\Sprache{Deutsch}

\Modulverantwortlicher{\Prof{Dr. Frank Kargl}}

\Moduldauer{1}

\Turnus{
\periodisch{\WiSe}
}

\Dozenten{
\Prof{Dr. Frank Kargl}
\Prof{Dr.-Ing. Franz J. Hauck} 
}

\Einordnung{
\ET{\Ba}{\Nebenfach}{Informatik}
}

\VoraussetzungenInhaltlich{Programmiererfahrung in Java, Modul Praktische
Informatik, Modul Programmieren von Systemen}
\VoraussetzungenFormal{Keine}

\Lernziele{Studierende können die Aufgaben von Kommunikationsschichten anhand
des ISO/OSI-Modells benennen und am Beispiel des Internets erläutern. Sie sind
in der Lage auf Basis von UDP und TCP kommunizierende Anwendungen in Java zu
entwickeln. Sie verstehen gängige Routingalgorithmen, Verfahren zur
zuverlässigen Datenübertragung und Protokolle zum Medienzugang und sind in der
Lage anhand ihrer Merkmale und Funktionen zu bewerten. Sie können skizzieren wie
grundlegende Verfahren der Computersicherheit funktionieren und wie diese auf
netzwerkbasierte Kommunikation anwendbar sind.}

\Inhalt{Anhand des Internets werden die Anwendungs-, Transport-, Netzwerk- und
Sicherungsschicht des ISO/OSI-Modells ausführlich behandelt. In den Übungen wird
dabei der Umgang mit der Netzwerk-API (Sockets) geübt und einzelne Funktionen
der Schichten in abgegrenzten Aufgaben nachgebildet. Schwerpunkte der Vorlesung
bilden Verfahren zur zuverlässigen Datenübertragung, zum Routing und zum
Medienzugang. Es werden die Protokolle HTTP, SMTP, POP3, DNS, TCP, UDP, IP, ARP,
ICMP und PPP behandelt. Auf der Ebene der Sicherungsschicht wird unter anderem
Ethernet betrachtet. Eine Einführung in die kryptographischen Verfahren der
Computersicherheit sowie deren Anwendung im Netzwerkbereich schließen den Stoff
ab.}

\Literatur{
\buch{J. F. Kurose, K. W. Ross: \textit{Computer Networking, A Top-Down
Approach.} 6th Ed., Addison-Wesley, 2012.}
\buch{J. F. Kurose, K. W. Ross: \textit{Computernetzwerke, Der Top-Down-Ansatz.}
5. Aufl., Pearson, 2012.}
}

\Grundlagen{--}

\Lehrformen{
  \Vlg{"`Grundlagen der Rechnernetze"', 2 SWS, WiSe}{Prof. Dr. Frank Kargl}
  \Ubg{Grundlagen der Rechnernetze: Vertiefung der Vorlesungsinhalte mittels Bearbeitung theoretischer und praktischer Aufgaben, 1 SWS, WiSe}{Dipl.-Inf. Benjamin Erb}
}

\Arbeitsaufwand{
\Praesenzzeit{45}    
\VorNachbereitung{75}
\Summe{120}
}

\Leistungsnachweis{schriftliche Prüfung am Ende des Semesters; keine
Leistungsnachweise; Notenbonus bei erfolgreichem Abschluss der Übungen}

\Notenbildung{Note der Modulprüfung}


