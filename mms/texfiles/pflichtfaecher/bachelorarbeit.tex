\svnid{$Id$}

\Modultitel{Bachelorarbeit}

\EnglischerTitel{Bachelor's Thesis}

\Modulkuerzel{80000}

\Sprache{Deutsch}

\SWS{0}

\ECTS{12}

\Moduldauer{1}

\Turnus{
  \periodisch{\Se}
}

\Modulverantwortlicher{
  \StudiendekanInf
}

\Dozenten{
  \alleDoz{Erstbetreuer der Bachelorarbeit}
}

\Einordnung{
  \Inf{\Ba}{\Abschlussarbeit}{Bachelorarbeit}
  \MedInf{\Ba}{\Abschlussarbeit}{Bachelorarbeit}
  \SwEng{\Ba}{\Abschlussarbeit}{Bachelorarbeit}
}

\VoraussetzungenInhaltlich{
Mindestens die Module der Pflichtfächer. Wünschenswert ist es, im Schwerpunkt grundlegende Module aus dem geplanten Gebiet der Bachelorarbeit belegt zu haben.
}

\VoraussetzungenFormal{
  Keine
}

\Lernziele{
Die Bachelorarbeit dient dazu, eine komplexe Problemstellung aus dem Gebiet der Informatik selbstständig unter Anwendung des Methodenwissens der Informatik zu bearbeiten und gemäß wissenschaftlicher Standards zu dokumentieren. Die Aufgabe einer Bachelorarbeit kann beispielsweise die Entwicklung von Software, Hardware, eine Beweisführung oder eine Literaturrecherche umfassen.
}

\Inhalt{
  Abhängig von der konkreten Themenstellung.
}

\Literatur{
  Abhängig von der konkreten Themenstellung.
}

\Grundlagen{--}

\Lehrformen{
  \BaArb{Wahl eines geeigneten Themas an einem der Institute der
  Informatik}{Dozenten der Informatik}
}

\Arbeitsaufwand{
  \Praesenzzeit{10}
  \VorNachbereitung{350}
  \Summe{360}
}

\Leistungsnachweis{
  Schriftliche Ausarbeitung und Abschlussvortrag.
}

\Notenbildung{
  Die Modulnote wird gemäß Prüfungsordnung gebildet.
}


