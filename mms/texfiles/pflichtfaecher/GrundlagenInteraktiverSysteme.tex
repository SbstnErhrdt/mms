\svnid{$Id$}

\Modultitel{Grundlagen Interaktiver Systeme}

\Modulkuerzel{71221}

\EnglischerTitel{Fundamentals in Interactive Systems}

\SWS{3}

\ECTS{6} 

\Sprache{Deutsch}

\Modulverantwortlicher{\StudiendekanInf}

\Moduldauer{1}

\Turnus{
% muss diskutiert werden
\periodisch{\WiSe}
}

\Dozenten{
\Prof{Dr. Heiko Neumann} 
}

\Einordnung{
% \Inf{\Ma}{Fortgeschrittene Methoden der Mathematik und Informatik in der Medizin}
\MedInf{\Ba}{\Pflichtfach}{\MEI}
\Inf{\Ba}{\Schwerpunkt}{}
\SwEng{\Ba}{\Pflichtfach}{\SoftwareEngineering}
}

\VoraussetzungenInhaltlich{Keine}
\VoraussetzungenFormal{Keine}

\Lernziele{
Die Studierenden sollen grundlegende Kenntnisse zum Entwurf interaktiver Systeme 
(insbesondere Software-Systeme) erwerben (Fachkompetenzen). Durch die Diskussion 
von Aspekten der Wahrnehmung und der Kognitionswissenschaften, der Betrachtung 
von Techniken unterschiedlicher Generationen von Interaktionsmetaphern sowie die 
Anwendung verschiedener formaler Methoden zur Notation und Analyse von 
Interaktionsmechanismen werden Studierende bef\"ahigt, bestehende L\"osungen zu 
bewerten und neue Ans\"atze f\"ur die Schnittstellenentwurf zu entwickeln 
(Methodenkompetenz).
}

\Inhalt{
\spiegelstrich{Einf\"uhrung}
\spiegelstrich{Wahrnehmung}
\spiegelstrich{Kognitive Prozesse und Interaktion}
\spiegelstrich{Empirische Aspekte zum Entwurf interaktiver Systeme}
\spiegelstrich{Die Schnittstelle - Ein- und Ausgabe-Ger\"ate in der MCI}
\spiegelstrich{Schnittstellenentwurf - Dialogform und Schnittstellenformate}
\spiegelstrich{Dialognotation - Graphen, Netze und graphische Methoden}
\spiegelstrich{Strukturierter Entwurf und Analyse interaktiver Systeme}
}

\Literatur{
Folgende Literatur hat Referenzcharakter f\"ur dieses Modul. Angaben zu spezieller 
und vertiefender Literatur erfolgen zu Beginn der Veranstaltung:
\buch{A. Dix, J. Finlay, G.Abowd, R. Beale: Human-Computer Interaction. Prentice-Hall, 1998}
\buch{J. Raskin: The Humane Interface. Addison-Wesley, 2000}
\buch{D. Benyon: Designing Interactive Systems. Pearson Education Ltd, 2010}
}

\Lehrformen{
\Vlg{Grundlagen Interaktiver Systeme (2 SWS)}{Prof. Heiko Neumann}
\Ubg{zu Grundlagen Interaktiver Systeme (1 SWS)}{Dipl.-Inform. Stephan Tschechne, Bj\"orn Wiedersheim}
In der Vorlesung werden Inhalte mittels digitaler Folienmaterialien vermittelt 
und anhand zus\"atzlicher Materialien detailliert. Die \"Ubungen werden begleitend zu den 
Vorlesungsinhalten gestaltet und beinhalten prim\"ar praktische Aufgaben zur 
Vertiefung der Inhalte.
}

\Arbeitsaufwand{
\Praesenzzeit{45}    
\VorNachbereitung{75}
\Summe{120}
}

\Leistungsnachweis{
Es findet eine Modulpr\"ufung f\"ur die Vorlesung statt, die erfolgreiche Bearbeitung 
der \"Ubungsaufgaben wird als Lernfortschrittskontrolle protokolliert. Das Erreichen 
einer Mindestanzahl an Punkten erzielt einen Notenbonus, der das Ergebnis der Pr\"ufung 
bis zur n\"achst besseren Zwischennote anhebt (die genauen Modalit\"aten hierzu werden 
zu Beginn der Veranstaltung mitgeteilt). Die Modulpr\"ufung (\"uber die Inhalte von 
Vorlesung und \"Ubungen) erfolgt schriftlich.
}

\Notenbildung{Die Modulnote ergibt sich aus der Modulprüfung.
Bei erfolgreicher Teilnahme an den Übungen wird dem Studierenden ein Notenbonus gemäß \S17 Abs. 3a der Rahmenordnung gewährt.
}

\Grundlagen{
Modul User Interface Technnologien
}


