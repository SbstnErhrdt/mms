\svnid{$Id$}

\Modultitel{User Interface Softwaretechnologie}

\Modulkuerzel{72027}

\EnglischerTitel{User Interface Softwaretechnologie}

\SWS{3}

\ECTS{4}

\Sprache{Deutsch}

\Moduldauer{1}

\Turnus{
\periodisch{\SoSe}
}

\Modulverantwortlicher{\StudiendekanInf}

\Dozenten{
\Prof{Dr. Michael Weber}
}

\Einordnung{
\MedInf{\Ba}{\Pflichtfach}{\MEI}
\SwEng{\Ba}{\Pflichtfach}{\SoftwareEngineering}
}

\VoraussetzungenInhaltlich{Modul Programmieren von Systemen, Modul Grundlagen Interaktiver Systeme}
\VoraussetzungenFormal{Keine}

\Lernziele{Die Teilnehmer sind in der Lage ereignis-basierte grafische
interaktive Systeme zu konzipieren und zu realisieren. Sie haben interne
Kenntnisse zu Fenstersystemen. Der Ansatz modell-basierter Entwicklung
interaktiver Systeme für verschiedene Modalitäten und verschiedene Geräte auf
verschiedenen Abstraktionsebenen wird beherrscht. Der Entwicklungszyklus des
User-Centered Design ist bekannt inklusive einfacher Evaluationsmethoden.

Diese Veranstaltung komplementiert die Veranstaltung "`Grundlagen interaktiver
Systeme"'.}

\Inhalt{
\spiegelstrich{Einführung in User Interface Softwaretechnologie}
\spiegelstrich{Präsentation – Fenster und Zeichnen}
\spiegelstrich{Ereignisbehandlung}
\spiegelstrich{Widgets und Layoutmanagement}
\spiegelstrich{Modell-basierte Entwicklung interaktiver Systeme}
\spiegelstrich{Prozesse zum Entwurf interaktiver Systeme}
}

\Literatur{
\buch {Dan R. Olsen, Jr.: Building Interactive Systems. Course Technology, 2009}
\buch {Alan Dix, Janet Finlay, Gregory Abowd, Russell Beale: Human-Computer Interaction, 3rd Edition, Prentice Hall, 2004}
\buch {Dan R. Olsen, Jr.: Developing User Interfaces. Morgan Kaufmann Publishers, 1998}
\buch {James D. Foley, Andries van Dam, Steven K. Feiner, John F. Hughes: Computer Graphics. 2nd Edition, Addison-Wesley, 1996}
\buch {Ben Shneiderman, Catherine Plaisant: Designing the User Interface. 5th Edition, Pearson, 2010}
\buch {Markus Dahm: Grundlagen der Mensch-Computer-Interaktion. Pearson Studium, 2006}
\buch {Bernhard Preim, Raimund Dachselt: Interaktive Systeme. Band 1, 2. Auflage, Springer, 2010}
}

\Grundlagen{--}

\Lehrformen{
\Vlg{User Interface Softwaretechnologie, 2 SWS}{Prof. Dr. Michael Weber}
\Ubg{User Interface Softwaretechnologie, 1 SWS}{Dipl.-Ing. Björn Wiedersheim, Dipl.-Ing. Stephan Tschechne}
}

\Arbeitsaufwand{
\Praesenzzeit{45}    
\VorNachbereitung{75}
\Summe{120}
}

\Leistungsnachweis{Die Modulprüfung erfolgt schriftlich.}

\Notenbildung{Die Modulnote ergibt sich aus der Modulprüfung.
Bei erfolgreicher Teilnahme an den Übungen wird dem Studierenden ein Notenbonus gemäß \S13 (5) der fachspezifischen Prüfungsordnung Informatik/Medieninformatik/Software Engineering gewährt.
}

%\Grundlagen{}

