\svnid{$Id$}

\Modultitel{Grundlagen der Gestaltung II}

\Modulkuerzel{72026}

\EnglischerTitel{Fundamentals of Design II}

\SWS{4}

\ECTS{6} 

\Sprache{Deutsch}

\Modulverantwortlicher{\StudiendekanInf}

\Moduldauer{1}

\Turnus{
\periodisch{\SoSe}
}

\Dozenten{
\LB{Roger Walk}
\LB{Roland Barth}
}
% Schwerpunkt (Bachelor)
% Kernfächer: Praktische Informatik, Angewandte Informatik, Technische Informatik, Theoretische Informatik
% Vertiefungsfach (Master)
% Projektfach (Master)
%    Ein Projektmodul darf ausschließlich Veranstaltungen der Form eines
%    Projektes, Laborpraktikums oder eines Seminars beinhalten und umfasst immer 16 LP
\Einordnung{
\MedInf{\Ba}{\Pflichtfach}{\MEI}
}

\VoraussetzungenInhaltlich{Keine}
\VoraussetzungenFormal{Modul Grundlagen der Gestaltung I}

\Lernziele{
Die Studierenden sind in der Lage selbsterklärende, leicht verständliche grafische interaktive Benutzerschnittstellen zu gestalten.
Komplexere Projekte können mittels systematischer Entwurfsmethoden gelöst werden.}

\Inhalt{
\spiegelstrich{Gestaltung von Benutzerschnittstellen für Interaktive Systeme}
\spiegelstrich{Funktionale und formal-ästhetische Entwurfspraxis}
\spiegelstrich{Systematische Entwurfsmethoden}
}

\Literatur{
\buch{Otl Aicher \& Martin Krampen: Zeichensysteme der visuellen Kommunikation,
Stgt, 1977}
\buch{Gui Bonsiepe: Interface. Design neu begreifen, Mannheim, 1996}
\buch{Alan Cooper: About Face 3.0: The Essentials of Interaction Design, Wiley
\& Sons, 2007}
\buch{Markus Dahm: Grundlagen der Mensch-Computer-Interaktion, Pearson, 2006}
\buch{Adrian Frutiger: Der Mensch und seine Zeichen, Fourier, 1978}
\buch{Armin Hofmann: Methodik der Form und Bildgestaltung, Niggli, 1965}
\buch{Herbert Kapitzki: Programmiertes Gestalten, Dieter Gitzel, 1980}
\buch{William Lidwell: Universal principles of Design, Rockport, 2003}
\buch{John Maeda: Creative Code, Birkhäuser, 2004}
\buch{John Maeda: Maeda \@ Media, Thames \& Hudson, 2000}
\buch{John Maeda: Design by Numbers, Link: http://dbn.media.mit.edu, 1999}
\buch{Bernhard Preim: Interaktive Systeme. Band 1: Grundlagen, Graphical User Interfaces, Informationsvisualisierung. Springer, 2010}
\buch{Jef Raskin: Das intelligente Interface, Addison-Wesley, 2001}
\buch{Jenifer Tidwell: Designing Interfaces. Patterns for Effective Interaction Design, O’Reilly Media Inc., 2006}
}

\Grundlagen{--}

\Lehrformen{
\Vlg{Grundlagen der Gestaltung II}{Roger Walk}
\Ubg{Grundlagen der Gestaltung II}{Roger Walk}}

\Arbeitsaufwand{
\Praesenzzeit{120}    
\VorNachbereitung{60}
\Summe{180}
}

\Leistungsnachweis{Der Leistungsnachweis erfolgt durch eine Dokumentation der eigenen Implementierungen und Entwürfe als Belegarbeit.}

\Notenbildung{Das Modul ist unbenotet. Bei ausreichenden Leistungen wird ein Leistungsnachweis (Schein) vergeben.}



