\svnid{$Id$}

\Modultitel{Grundlagen der Rechnerarchitektur}

\Modulkuerzel{72008}

\ECTS{8}

\SWS{6} % 4V + 1Ü + 1P

\Sprache{Deutsch}

\Turnus{
  \periodisch{\WiSe}
}

\Moduldauer{1}

\Modulverantwortlicher{\StudiendekanInf}

\Dozenten{
  \Prof{Dr. Heiko Falk}
  \Prof{Dr.-Ing. Franz J. Hauck}
  \Prof{Dr. Frank Kargl}
  \Prof{Dr.-Ing. Frank Slomka}
  \LB{Dipl.-Ing. Jörg Siedenburg}
}


% DONE: Medieninformatik Schwerpunkt mit rein.
\Einordnung{
  \Inf{\Ba}{\Pflichtfach}{\TSI}
  \Inf{\La}{\Wahlmodul}{}
  \IST{\Ba}{\Pflichtmodul}{}
  \MedInf{\Ba}{\Schwerpunkt}{Medieninformatik}
  \SwEng{\Ba}{\Pflichtfach}{\TSI}
}

\VoraussetzungenInhaltlich{
  Grundlagenkenntnisse der technischen Informatik, wie sie im Rahmen des Bachelor-Moduls "`Grundlagen der Betriebssysteme und Rechnernetze"' vermittelt werden, sind von Vorteil. Die relevanten Grundlagen werden für Quereinsteiger nochmals kurz rekapituliert.
}

\Lernziele{
  Die Studierenden identifizieren die Grundlagen der Funktionsweise von Rechensystemen aus der Sicht der internen Rechnerarchitektur. Sie fassen ein Rechensystem als Ausführungsplattform von Software auf, wie es aus der Perspektive des Architekten wahrgenommen wird, d.h. sie erkennen die interne Struktur und den physischen Aufbau von Rechensystemen. Die Studierenden können analysieren, wie hochspezifische und individuelle Rechner aus einer Sammlung gängiger Einzelkomponenten zusammengesetzt werden. Sie sind in der Lage, die unterschiedlichen Abstraktionsebenen heutiger Rechensysteme - von Gattern und Schaltungen bis hin zu Prozessoren und deren systemnaher Programmierung - zu unterscheiden und zu erklären.

  Nach erfolgreichem Besuch der Veranstaltung sind die Studierenden in der Lage, die Wechselwirkungen zwischen einem physischen Rechensystem und der darauf ausgeführten Software beurteilen zu können. Insbesondere sollen sie die Konsequenzen der Ausführung von Software in den hardwarenahen Schichten von der Assemblersprache bis zu Gattern erkennen können. Sie sollen so in die Lage versetzt werden, Auswirkungen unterer Schichten auf die Leistung des Gesamtsystems abzuschätzen und geeignete Optionen vorzuschlagen.
}

% DONE: Inhalte anhand jetziger Vorlesungs-Gliederung aktualisiert.
\Inhalt{
  \spiegelstrich{Einführung: Ausführungsplattformen, Aufbau heutiger Rechner}
  \spiegelstrich{Kombinatorische Logik: Gatter, Boolesche Algebra, Schaltfunktionen, Synthese von Schaltungen, Schaltnetze}
  \spiegelstrich{Sequentielle Logik: Flip-Flops, Schaltwerke, systematischer Schaltwerkentwurf}
  \spiegelstrich{Technologische Grundlagen: Halbleiter-Bauelemente, Programmierbare Logikbausteine}
  \spiegelstrich{Rechnerarithmetik: Ganzzahlige Addition, Subtraktion, Multiplikation und Division, BCD-Arithmetik}
}
\InhaltForts{
  \spiegelstrich{Grundlagen der Rechnerarchitektur: Grundbegriffe, Programmiermodelle, MIPS-Einzelzyklusmaschine, Pipelining, Dynamisches Scheduling}
  \spiegelstrich{Speicher-Hardware: Speicherhierarchien, SRAM, DRAM, Translation Look-Aside Buffer, Caches, Massenspeicher}
  \spiegelstrich{Ein-/Ausgabe: Ein-/Ausgabe aus Sicht der CPU, Prinzipien der Datenübergabe, Point-to-Point Verbindungen, Busse, CRC-Zeichen}
}

\Literatur{
  \buch{A. Clements. The Principles of Computer Hardware. 3. Auflage, Oxford University Press, 2000.}
  \buch{A. Tanenbaum, J. Goodman. Computerarchitektur. Pearson, 2001.}
  \buch{D. Patterson, J. Hennessy. Rechnerorganisation und -entwurf. Elsevier, 2005.}
}

\Grundlagen{
  Weiterführende Module aus den verteilten Systemen, der Robotik und den eingebetteten Systemen.
}

% DONE: Benji raus, Nico rein.
\Lehrformen{
  \Vlg{Grundlagen der Rechnerarchitektur, 4 SWS}{Prof. Dr. Heiko Falk}
  \Ubg{Grundlagen der Rechnerarchitektur, 1 SWS}{Dipl.-Inf. Nicolas Roeser}
  \Lab{Grundlagen der Rechnerarchitektur, 1 SWS}{Dipl.-Ing. Jörg Siedenburg}
}

\Arbeitsaufwand{
  \Praesenzzeit{90}
  \VorNachbereitung{150}
  \Summe{240}
}

% DONE: Praktikum in Labor umbenannt.
% DONE: Orientierungsprüfung ersatzlos gestrichen, da nicht in §5 FSPO erwähnt!
\Leistungsnachweis{
  Die Vergabe der Leistungspunkte erfolgt aufgrund des Bestehens der schriftlichen Modulprüfung und der erfolgreichen Teilnahme am Labor "`Grundlagen der Rechnerarchitektur"'.
}

\VoraussetzungenFormal{
}

\Notenbildung{Die Modulnote ergibt sich aus der Modulprüfung.
Bei erfolgreicher Teilnahme an den Übungen wird dem Studierenden ein Notenbonus gemäß %$17 Abs. 3a der Rahmenordnung gewährt.
}
