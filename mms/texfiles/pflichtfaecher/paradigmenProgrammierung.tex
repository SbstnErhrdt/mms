\svnid{$Id$}


\Modultitel{Paradigmen der Programmierung}

\Modulkuerzel{72006}

\EnglischerTitel{Programming Paradigms}

\SWS{
3
}

\ECTS{
4}

\Sprache{
Deutsch, Englisch (auf Wunsch)
}

\Moduldauer{1} %Hier steht ob das Modul ein oder zwei Semester lang ist.

\Turnus{
\periodisch{\SoSe}
}

\Einordnung{
  \Inf{\Ba}{\Pflichtfach}{\PAI}
  \MedInf{\Ba}{\Pflichtfach}{\PAI}
  \SwEng{\Ba}{\Pflichtfach}{\PAI}
}

\VoraussetzungenInhaltlich{Modul Einführung in die Informatik}



\Lernziele{
Ziel des Moduls ist, die Studierenden in Erweiterung des im Modul 
'Einführung in die Informatik' gelernten vertraut zu machen mit 
Programmierparadigmen und -stilen, die Alternativen und 
Ergänzungen zur prozedural/imperativen Programmierung darstellen.

Die Studierenden sollen grundlegendes Verständnis und Kenntnisse
über Prinzipien und Verfahren der deklarativen Programmierung erhalten,
können sie anwenden und besitzen entsprechende Programmiererfahrung.



}


\Inhalt{
\spiegelstrich{Die Vorlesung führt in alternative, deklarative Programmierkonzepte 
ein durch Vermittlung und Einübung geeigneter Sprachen für das funktionale (z.B. Haskell) und 
logisch-relationale (z.B. Prolog, CHR) Programmieren.

Die unterschiedlichen Ansätze werden miteinander vergleichbar gemacht und in Beziehung gesetzt.}

\spiegelstrich{Die Übung ermöglicht es, praktische Erfahrungen mit deklarativen Programmiersprachen zu sammeln.
Die Übungen beinhalten sowohl theoretische als auch programmier-praktische Aufgaben.}
}


\Literatur{ 
	%\buch{Albert Einstein: Allgemeine Relativitätstheorie, Vieweg 1936}
	%\aufsatz{Albert Einstein: Zur Elektrodynamik bewegter Körper, Annalen der Physik 1905}
\skript{Vorlesungsfolien}
\buch{W. F. Clocksin, C. S. Mellish: Programming in Prolog, 5. Auflage, Springer 2003}
\buch{M. Lipovaca, Learn you a Haskell for Great Good!, No Starch Press 2011, http://learnyouahaskell.com}
}


\Lehrformen{
\Vlg{Paradigmen Programmierung}{Prof. Dr. Dipl.-Ing. Thomas Frühwirth, Dr. Alexander Raschke}
\Ubg{Paradigmen Programmierung}{Prof. Dr. Dipl.-Ing. Thomas Frühwirth, Dr. Alexander Raschke}

Im Rahmen der Vorlesung werden Inhalte mittels Folien und Tafel vermittelt. 
Die Übungen beinhalten sowohl theoretische als auch programmier-praktische Aufgaben.
}


\Arbeitsaufwand{%Hier soll einheitlich die Präsenzzeit angegeben werden. Dabei wird kein
               %Unterschied zwischen Sommersemester und Wintersemester gemacht. 
               \Praesenzzeit{30}    % 15 * SWS
               \VorNachbereitung{90} % SEMINAR Literaturstudium, schriftliche Ausarbeitung und Abschlusspräsentation
               \Summe{120}
}


\Leistungsnachweis{
Die Modulprüfung erfolgt schriftlich. Die genauen Modalitäten werden zu Beginn der Veranstaltung bekannt gegeben.
}


\Notenbildung{Die Modulnote ergibt sich aus der Modulprüfung.
Bei erfolgreicher Teilnahme an den Übungen wird dem Studierenden ein Notenbonus gemäß \S13 (5) der fachspezifischen Prüfungsordnung Informatik/Medieninformatik/Software Engineering gewährt.
}

\Dozenten{%
  \Prof{Dr. Dipl.-Ing. Thomas Frühwirth}
  \LB{Dr. Alexander Raschke}
}

\Modulverantwortlicher{%
\StudiendekanInf
}


\VoraussetzungenFormal{
Keine.
}

\Grundlagen{Modul Informationssysteme
}

\Ilias{
}
