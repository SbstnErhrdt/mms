\svnid{$Id$}

\Modultitel{Softwaregrundprojekt}

\EnglischerTitel{Software Construction}

\Modulkuerzel{72007}

\Sprache{Deutsch}

\SWS{6}

\ECTS{10}

\Moduldauer{2}

\Turnus{
  \periodisch{\WiSe}
}

\Modulverantwortlicher{
\StudiendekanInf
}

\Dozenten{
  \Prof{Dr. Helmuth Partsch}
  \LB{Alexander Nassal}
}

\Einordnung{
  \Inf{\Ba}{\Pflichtfach}{\PAI}
  \MedInf{\Ba}{\Pflichtfach}{\PAI}
  \SwEng{\Ba}{\Pflichtfach}{\PAI}
  \Inf{\La}{\Pflichtmodul}{}
}

\VoraussetzungenInhaltlich{
  Die Beherrschung objektorientierter Programmierung und grundlegende Datenbankkenntnisse werden vorausgesetzt.
  Das begleitende Modul Softwaretechnik wird vorausgesetzt.
}

\VoraussetzungenFormal{
  Keine
}

\Lernziele{
Die Studierenden sollen die wesentlichen Aspekte des Software Engineering praktisch kennen und beherrschen lernen. Dazu gehören vor allem
\spiegelstrich{Bedeutung, Schwierigkeiten und Möglichkeiten des Software Engineering kennen und beschreiben können}
\spiegelstrich{einschlägige Kenntnisse über Software, Softwareentwicklung, Softwarequalität und Projektmanagement im Rahmen einer konkreten Problemstellung praktisch anwenden können}
\spiegelstrich{aus eigener Erfahrung argumentieren können, dass erfolgreiches Software Engineering sorgfältige Planung, systematische Vorgehensweise und Disziplin erfordert und dass gründliche und systematische Anforderungsanalyse sowie sorgfältiger Grob- und Feinentwurf unabdingbar für den Erfolg eines Softwareprojekts sind}
\spiegelstrich{Software-Entwicklungswerkzeuge kennen und damit umgehen können}
\spiegelstrich{in der Lage sein, gängige Qualitätssicherungsmaßnahmen, vor allem Test und Reviews, sinnvoll einzuplanen und umzusetzen}
\spiegelstrich{erfahren, welche nicht-fachlichen Schwierigkeiten (Zeitökonomie, Termindruck, Kommunikations- und Abstimmungsprobleme, Schwierigkeiten in der Zusammenarbeit mit anderen) im Rahmen der Softwareerstellung auftreten können und wie man erfolgreich damit umgeht}
\spiegelstrich{Teamarbeit, Präsentationstechniken, schriftliche Dokumentation und Techniken der Projektabwicklung aus eigener praktischer Erfahrung kennen}
}

\Inhalt{
Das Softwaregrundprojekt ist eine Pflichtveranstaltung für Studierende in den Bachelor-Studiengängen Informatik, Medieninformatik, Softwareengineering und Informationssystemtechnik sowie Lehramt Informatik. In diesem Projekt sollen die in vorangegangenen Lehrveranstaltungen erlernten Fähigkeiten bei der praxisnahen Abwicklung eines umfangreichen Softwareprojekts angewendet werden. Die Projektinhalte stammen überwiegend aus dem Bereich der Universitätsverwaltung. Es handelt sich dabei um reale Aufgabenstellungen die für jeden Durchgang des  Softwaregrundprojekts einzigartig sind.
Schwerpunkt dabei ist die methodische Softwareerstellung an Hand eines vorgegebenen Prozesses. Unter Anwendung der Fusionmethode werden die Phasen Anforderungsanalyse (zur Spezifikation der Anforderungen), Entwurf der Softwarearchitektur, Implementierung und Qualitätssicherung in etwa gleichem Umfang (und entsprechendem Arbeitsaufwand) durchgeführt. Alle Artefakte der entwickelten Software und des dazugehörigen Softwareentwicklungsprozesses werden umfangreich dokumentiert. Dazu wird vor allem UML2 verwendet. Das Softwaregrundprojekt wird ausschließlich im Team absolviert.
}

\Literatur{
  \skript{Kopien der Folien der Begleitvorlesung}
}

\Lehrformen{
  \Prj{Softwaregrundprojekt 1}{Prof. Dr. Helmuth Partsch, Alexander Nassal}
  \Prj{Softwaregrundprojekt 2}{Prof. Dr. Helmuth Partsch, Alexander Nassal}
}

\Arbeitsaufwand{
  \Praesenzzeit{90}
  \VorNachbereitung{210}
  \Summe{300}
}

\Leistungsnachweis{
  Die Vergabe der Leistungspunkte für die erfolgreiche Teilnahme am Projekt erfolgt aufgrund der erfolgreichen Absolvierung aller Projektphasen.
}

\Notenbildung{
  Das Projekt ist unbenotet.
  Für Lehramtstudierende ist das Softwaregrundprojekt aufgrund der landesweit gültigen Prüfungsordnung benotet
}

\Grundlagen{
  Modul Anwendungsprojekt Software-Engineering
}


