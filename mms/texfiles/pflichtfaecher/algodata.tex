\svnid{$Id$}

\Modultitel{Algorithmen und Datenstrukturen}

\Modulkuerzel{70318}

\EnglischerTitel{Algorithms and Data Structures}

\SWS{6}

\ECTS{8} 

\Sprache{Deutsch}

\Modulverantwortlicher{\StudiendekanInf}

\Moduldauer{1}

\Turnus{
\periodisch{\WiSe}
}

\Dozenten{
\Prof{Dr. Uwe Sch\"oning}
\Prof{Dr. Jacobo Tor\'an}
\Prof{Dr. Enno Ohlebusch}
}
\Einordnung{
\Inf{\Ba}{\Pflichtfach}{\TMI}
\MedInf{\Ba}{\Pflichtfach}{\TMI}
\SwEng{\Ba}{\Pflichtfach}{\TMI}
\Inf{\La}{\Pflichtmodul}{\TMI}
\ET{\Ba}{\Wahlpflichtmodul}{}\\
\IST{\Ba}{\Wahlpflichtmodul}{}
}

\VoraussetzungenInhaltlich{Modul Einführung in die Informatik, Modul Formale Grundlagen}
\VoraussetzungenFormal{Keine}

\Lernziele{Die Studierenden erwerben fundierte Kenntnisse zum Erstellen
und Analysieren von Algorithmen f\"ur verschiedene praktische Anwendungen
sowie die hierzu vorteilhaften Datenstrukturen.
Sie verstehen die verschiedenen algorithmischen Problemtypen den unterschiedlichen 
Algorithmenparadigmen zuzuordnen. F\"ur jedes betrachtete Algorithmenparadigma sind
sie mit der zugrunde liegenden formalen Analyse vertraut und wissen diese
anzuwenden und nach deren Effizienz bzw. Komplexit\"at einzuordnen. 
Die Studierenden sind in der Lage, aus  Problemspezifikationen
geeignete Datenstrukturen zu deren Repr\"asentation und zur Unterst\"utzung ihrer algorithmischen
L\"osung zu entwerfen. 
}

\Inhalt{
Im Modul werden Begriffe, Methoden und Resultate aus dem Bereich der Algorithmen und 
Datenstrukturen vorgestellt, die in verschiedenen Gebieten der Informatik Anwendung 
finden.
\spiegelstrich{Asymptotische Notationen f\"ur die Absch\"atzung von Worst-Case oder Average-Case
Laufzeiten.}
\spiegelstrich{Analyse rekursiver Algorithmen und der dabei entstehenden Rekursionsgleichungen, Mastertheorem.}
\spiegelstrich{Verschiedene elementare und fortgeschrittene Sortier- und Selektionsverfahren und ihre Analyse. Informationstheoretische
untere Schranke f\"ur Sortieren.}
}
\InhaltForts{
\spiegelstrich{Hashing, Geburtstagsproblem, Kollisionsstrategien.}
\spiegelstrich{Das Algorithmenprinzip Dynamisches Programmieren mit entsprechenden Beispielen.}
\spiegelstrich{Das Algorithmenprinzip Greedy mit entsprechenden Beispielen. Matroide.}
\spiegelstrich{Algorithmen auf Graphen: Dijkstra-, Kruskal-, Warshall-Algorithmus.}
\spiegelstrich{Algebraische und zahlentheoretische Algorithmen.}
\spiegelstrich{Algorithmen f\"ur das (String-) Matching.}
\spiegelstrich{Optimierung von B\"aumen, Branch-and-Bound, heuristische Verfahren.}

}

\Literatur{
\skript{Vorlesungsskript}
\buch{U. Sch\"oning: Algorithmik, Spektrum Verlag, Nachdruck 2011}
\buch{T.H. Cormen, C.E. Leiserson, R.L. Rivest, C. Stein: Introduction
to Algorithms. Second Edition. The MIT Press, 2001.}

}

\Lehrformen{
\Vlg{Algorithmen und Datenstrukturen, 4 SWS}{}
\Ubg{Algorithmen und Datenstrukturen, 2 SWS}{}
}

\Arbeitsaufwand{
\Praesenzzeit{90}    
\VorNachbereitung{150}
\Summe{240}
}

\Leistungsnachweis{Die Modulpr\"ufung erfolgt schriftlich.}

\Notenbildung{Die Modulnote ergibt sich aus der Modulprüfung.
Bei erfolgreicher Teilnahme an den Übungen wird dem Studierenden ein Notenbonus gemäß \S13 (5) der fachspezifischen Prüfungsordnung Informatik/Medieninformatik/Software Engineering gewährt.
}

\Grundlagen{Modul Logik, Berechenbarkeit und Komplexit\"at und Informationssysteme}
