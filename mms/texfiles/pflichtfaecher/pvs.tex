\svnid{$Id$}

%\input{header.tex}

\Modultitel{Programmierung von Systemen}

\Modulkuerzel{71365}

\EnglischerTitel{Programming of Systems}

\SWS{
%An dieser Stelle werden die Präsenzzeiten angegeben (V/Ü)
%2+2 (V/Ü)
6
}

\ECTS{
%Anzahl der Leistungspunkte
8}

\Sprache{
%Unterrichtssprache
Deutsch
}

\Moduldauer{1} %Hier steht ob das Modl ein oder zwei Semester lang ist.

\Turnus{
%\sporadisch{\WiSe 2010}
\periodisch{\SoSe}
%\periodisch{\SoSe}
}

\Dozenten{
  %Ein Hochschullehrer, Honorarprofessor, Privatdozent, Gastprofessor oder ein
  %Lehrbeauftragter gemäß §44 LHG Baden-Württemberg. Hier stehen alle Hochschullehrer, 
  %die am Modul mitwirken.
  \Prof{Dr. Franz Hauck}
  \Prof{Dr. Manfred Reichert}
}


\Modulverantwortlicher{
%Ein Dozent mit Prüfungsberechtigung, der ein Modul einrichtet und pflegt.
\StudiendekanInf
}

\Einordnung{
\Inf{\Ba}{\Pflichtfach}{\PAI}
\MedInf{\Ba}{\Pflichtfach}{\PAI}
\IST{\Ba}{\Pflichtmodul}{}
\SwEng{\Ba}{\Pflichtfach}{\PAI}
\Inf{\La}{\Pflichtmodul}{}
\ET{\Ba}{\Nebenfach}{Informatik}
}

\VoraussetzungenInhaltlich{
Modul Einführung in die Informatik
}

\VoraussetzungenFormal{
keine
}

\Lernziele{
Die Studierenden können Methoden und Werkzeuge der Programmierung, wie sie für die Entwicklung komplexer und interaktiver Software-Systeme (z.B. Oberflächenprogrammierung, Datenbankoperationen) notwendig sind, beschreiben und beurteilen. Dadurch sind sie in der Lage, eigenständig komplexe und interaktive Software-Systeme zu konzipieren und entwickeln. 
}

\Inhalt{
	%Hier soll ein Überbick über die fachlichen Inhalte gegeben werden.
	\spiegelstrich {Ereignisgesteuerte Programmierung}
	\spiegelstrich {Ausnahmebehandlung}
	\spiegelstrich {Programmierung graphischer und interaktiver Anwendungen}
	\spiegelstrich {Speicherung und Austausch von Anwendungsdaten mittels Dateien}
	\spiegelstrich {Modellierung und Anwendung relationaler Datenbanken
	(Datenbankentwurf, SQL, Relationenalgebra, Speicherstrukturen)} 
	\spiegelstrich {Modellierung und Programmierung nebenläufiger und verteilter Anwendungen}
	\spiegelstrich {Programmierumgebungen}
	\spiegelstrich {Methoden zum Softwareentwurf}
}

\Literatur{ 

%Für das Modul empfohlene Bücher oder Aufsätze.}
	\skript{Vorlesungsskript}
	\skript{Weiterführende Literatur wird in der Lehrveranstaltung bekannt gegeben.}
}

\Lehrformen{
\Vlg{Programmierung von Systemen}{4~SWS}
\Ubg{Programmierung von Systemen}{2~SWS}
%\Tut{Programmierung von Systemen}{Diverse}
% \Lab{Entwurfsmethodik Eingebetteter Systeme}{Dipl.-Ing. Tobias Bund}
% \Prj{Echtzeitkommunikationssysteme}{Dipl.-Inf. Steffen Moser}
% \Sem{Eingebettete Systeme}{Prof. Dr.-Ing. Frank Slomka}
% \ProSem{Eingebettete Systeme}{Dipl.-Inf. Victor Pollex}
}

\Arbeitsaufwand{%Hier soll einheitlich die Präsenzzeit angegeben werden. Dabei wird kein
               %Unterschied zwischen Sommersemester und Wintersemester gemacht. 
               \Praesenzzeit{90}    % 15 * SWS
               \VorNachbereitung{150}
               \Summe{240}
}

\Leistungsnachweis{
Die Modulprüfung erfolgt in Form einer schriftlichen Klausur.
}

\Notenbildung{Die Modulnote ergibt sich aus der Modulprüfung.
Bei erfolgreicher Teilnahme an den Übungen wird dem Studierenden ein Notenbonus gemäß \S13 (5) der fachspezifischen Prüfungsordnung Informatik/Medieninformatik/Software Engineering gewährt.
}

\Grundlagen{
Modul Softwaretechnik, Modul Softwaregrundprojekt und das Modul Informationssysteme
}

% \Ilias{
% Optional.
% }

