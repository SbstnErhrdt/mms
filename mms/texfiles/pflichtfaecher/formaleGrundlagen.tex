\svnid{$Id$}

%
\Modultitel{Formale Grundlagen}

\Modulkuerzel{70486}

\EnglischerTitel{Formal Foundations}

\SWS{6}

\ECTS{8} 

\Sprache{Deutsch}

\Modulverantwortlicher{\StudiendekanInf}

\Moduldauer{1}

\Turnus{
\periodisch{\WiSe}
}

\Dozenten{
\Prof{Dr. Uwe Sch\"oning}
\Prof{Dr. Jacobo Tor\'an}
\Prof{Dr. Enno Ohlebusch}
}
\Einordnung{
\Inf{\Ba}{\Pflichtfach}{\TMI}
\MedInf{\Ba}{\Pflichtfach}{\TMI}
\SwEng{\Ba}{\Pflichtfach}{\TMI}
\Inf{\La}{\Pflichtmodul}
}

\VoraussetzungenInhaltlich{Keine}
\VoraussetzungenFormal{Keine}

\Lernziele{Die Studierenden k\"onnen mit den in der Mathematik und Theoretischen
Informatik gebr\"auchlichen  Formalismen zur Beschreibung von Mengen, Mengensystemen, Folgen,
Alphabeten, W\"ortern sowie den einschl\"agigen Beweistechniken wie
direkter, indirekter Beweis, Induktionsbeweis, Strukturelle Induktion, Schubfachschlussprinzip
souver\"an umgehen und verstehen diese Methoden geeignet anzuwenden.
Sie sind mit den Einsatz und Nutzen von formalen Grammatiken, Automaten, Codes und
Booleschen Funktionen vertraut und wissen diese in 
ihrer Komplexit\"at einzuordnen.
}

\Inhalt{
Im Modul werden die notwendigen Grundbegriffe f\"ur den
Umgang mit der mathematisch-formalen Symbolik
wie Mengen, Folgen, Quantoren, Codes, Boole'sche Algebra sowei die hierzu 
notwendigen Beweistechniken behandelt. 
\spiegelstrich{Formalismen zur Beschreibung von Mengen, Mengensystemen, Folgen, Alphabeten, W\"ortern, Sprachen, Codes, Relationen, Funktionen, Permutationen sowie deren elementaren Eigenschaften.}
\spiegelstrich{Elementare Beweistechniken: direkter Beweis, indirekter Beweis,
Fallunterscheidung,  Induktionsbeweis, Abz\"ahlargument, Schubfachprinzip, Inklusions-Exklusionsprinzip, Existenz und Eindeutigkeit}
\spiegelstrich{Elemente der Codierungs- und Informationstheorie. Entropiebegriff.}
\spiegelstrich{Boole'sche Algebra, Boole'sche Funktionen, das Perzeptron, Schaltkreiskomplexit\"at}
\spiegelstrich{Formale Grammatiken und Automaten/Turingmaschinen 
und deren Eigenschaften. Chomsky-Hierarchie.}

}

\Literatur{
\skript{Vorlesungsskript}
\buch{U. Sch\"oning, H.A. Kestler: Mathe-Toolbox. Lehmanns Media, 2. erw. Auflage, 2011.}
\buch{U. Sch\"oning: Theoretische Informatik - kurz gefasst. 5. Auflage, Spektrum, 2008}
\buch{I. Wegener: Theoretische Informatik. Teubner, 1993.}
\buch{N. Blum: Einf\"uhrung in Formale Sprachen, Berechenbarkeit, Informations- und Lerntheorie. Oldenbourg, 2007.}

}

\Lehrformen{
\Vlg{Formale Grundlagen}{Prof. Dr. Uwe Sch\"oning}
\Ubg{Formale Grundlagen}{}
}

\Arbeitsaufwand{
\Praesenzzeit{90}    
\VorNachbereitung{150}
\Summe{240}
}

\Leistungsnachweis{Die Modulpr\"ufung erfolgt schriftlich.
Die Modulprüfung zählt als Orientierungsprüfung Formale Grundlagen nach §5 der fachspezifischen Prüfungsordnung Informatik, Medieninformatik und Softwareengineering.}

\Notenbildung{Die Modulnote ergibt sich aus der Modulprüfung.
Bei erfolgreicher Teilnahme an den Übungen wird dem Studierenden ein Notenbonus gemäß \S13 (5) der fachspezifischen Prüfungsordnung Informatik/Medieninformatik/Software Engineering gewährt.
}

\Grundlagen{
Modul Algorithmen und Datenstrukturen, wünschenswert ist es dieses Modul vor dem Besuch eines Seminars abgeschlossen zu haben.
}

