\svnid{$Id$}

\Modultitel{Medienpsychologie}

\Modulkuerzel{70062}

\EnglischerTitel{}

\SWS{3}

\ECTS{4} 

\Sprache{Deutsch}

\Modulverantwortlicher{
\StudiendekanInf}

\Moduldauer{1}

\Turnus{
\periodisch{\SoSe}
}

\Dozenten{
\Prof{Dr. Tina Seufert}
}

\Einordnung{
\MedInf{\Ba}{\Pflichtfach}{\MEI}
}

\VoraussetzungenInhaltlich{Keine}
\VoraussetzungenFormal{Keine}

\Lernziele{Die Studierenden sollen die zentralen Fragestellungen, die wesentlichen
theoretischen Modelle sowie wichtige Befunde der wissenschaftlichen Disziplinen
Medienpsychologie und Medienpädagogik kennen lernen. Sie sind in der Lage, die Theorien bei der Beurteilung und Gestaltung medial präsentierter Informationen umzusetzen.
Sie sollen zudem lernen, empirische Forschungsmethoden zu verstehen und in Bezug auf aktuelle mnedienychologische / medienpädagogische oder auch medieninformatische Fragen anzuwenden. 
}

\Inhalt{
Die Vorlesung gibt einen Überblick über die Disziplinen Medienpsychologie
und Medienpädagogik.  Es werden die wichtigsten theoretischen Ansätze vorgestellt und anhand von Beispielen erörtert und aktiv bearbeitet. Eine Schwerpunktsetzung erfolgt auf die medienpsychologischen bzw. medienpädagogischen Anwendungsfelder (z.B. Medienkompetenz,
Mediengestaltung). 
Ein zweiter Themenschwerpunkt der Vorlesung ist ein Überblick über empirische Forschungsmethoden. Studierende lernen hier die wichtigsten Grundbegriffe der Statistik und Testtheorie kennen, um später selbständig kleinere empirische Arbeiten, z.B. im Rahmen ihrer Bachelorarbeiten durchführen zu können.

}

\Literatur{
\skript{Dresel, M, Poguntke, M \& Schneider, M. (2007). Einführung in die
Forschungsmethoden der empirischen Sozialforschung. Lehrtext für Studierende der Informatik und ingenieurwissenschaftlicher Fächer. Vorlesungsskript Universität Ulm.}
\buch{Seufert, T.; Leutner, D. \& Brünken, R. (2004). Psychologische Grundlagen
des Lernens mit Neuen Medien. Lehrbrief für den Fernstudiengang Medien und Bildung der Universität Rostock.}
\buch{Mayer, R. E. (Ed.). (2005). The Cambridge Handbook of Multimedia Learning. New York: Cambridge University Press.}
}

\Grundlagen{--}

\Lehrformen{
\Vlg{Medienpsychologie / -pädagogik}{\Prof{Dr. Tina Seufert}}
}

\Arbeitsaufwand{
\Praesenzzeit{45}    
\VorNachbereitung{75}
\Summe{120}
}

\Leistungsnachweis{Die Vergabe der Leistungspunkte erfolgt aufgrund des Bestehens einer schriftlichen Prüfung. Die Anmeldung zu dieser Prüfung setzt keine Leistungsnachweise voraus.}

\Notenbildung{Die Modulnote entspricht dem Prüfungsergebnis.}
