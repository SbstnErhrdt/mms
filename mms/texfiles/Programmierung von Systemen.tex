\svnid{$Id$}

\Modultitel{Programmierung von Systemen}

\Modulkuerzel{71365}

\EnglischerTitel{Programming of Systems}

\SWS{6}

\ECTS{8}

\Sprache{Deutsch}

\Moduldauer{1}

\Turnus{
\periodisch{SoSe}
}

\Dozenten{
\Prof{Dr. Franz Hauck}
\Prof{Dr. Manfred Reichert}
}

\Modulverantwortlicher{\StudiendekanInf}

\Einordnung{
\Inf{\Ba}{\Pflichtfach}{\PAI}
\MedInf{\Ba}{\Pflichtfach}{\PAI}
\SwEng{\Ba}{\Pflichtfach}{\PAI}
\IST{\Ba}{\Pflichtmodul}{}
\Inf{\La}{\Pflichtmodul}{}
\ET{\Ba}{\Nebenfach}{Informatik}
}

\VoraussetzungenInhaltlich{
Modul Einführung in die Informatik}

\VoraussetzungenFormal{
keine}

\Lernziele{
Die Studierenden können Methoden und Werkzeuge der Programmierung, wie sie für die Entwicklung komplexer und interaktiver Software-Systeme (z.B. Oberflächenprogrammierung, Datenbankoperationen) notwendig sind, beschreiben und beurteilen. Dadurch sind sie in der Lage, eigenständig komplexe und interaktive Software-Systeme zu konzipieren und entwickeln.
}

\Inhalt{
\spiegelstrich{Ereignisgesteuerte Programmierung}
\spiegelstrich{Ausnahmebehandlung}
\spiegelstrich{Programmierung graphischer und interaktiver Anwendungen}
\spiegelstrich{Speicherung und Austausch von Anwendungsdaten mittels Dateien}
\spiegelstrich{Modellierung und Anwendung relationaler Datenbanken (Datenbankentwurf, SQL, Relationenalgebra, Speicherstrukturen)}
\spiegelstrich{Modellierung und Programmierung nebenläufiger und verteilter Anwendungen}
\spiegelstrich{Programmierumgebungen}
\spiegelstrich{Methoden zum Softwareentwurf}
}

\Literatur{
<ul><li>Skript: Vorlesungsskript</li> <li>Skript: Weiterführende Literatur wird in der Lehrveranstaltung bekannt gegeben.</li></ul>
}

\Lehrformen{
\Vlg{Programmierung von Systemen}{\Prof{4~SWS}}
\Ubg{Programmierung von Systemen}{\Prof{2~SWS}}
}

\Arbeitsaufwand{
\Praesenzzeit{90}
\VorNachbereitung{150}
\Summe{240}
}

\Leistungsnachweis{
Die Modulprüfung erfolgt in Form einer schriftlichen Klausur.
}

\Notenbildung{
Die Modulnote ergibt sich aus der Modulprüfung. Bei erfolgreicher Teilnahme an den Übungen wird dem Studierenden ein Notenbonus gemäß §13 (5) der fachspezifischen Prüfungsordnung Informatik/Medieninformatik/Software Engineering gewährt.
}

\Grundlagen{
Modul Softwaretechnik, Modul Softwaregrundprojekt und das Modul Informationssysteme
}