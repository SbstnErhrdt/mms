\svnid{$Id$}

\Modultitel{Grundlagen der Gestaltung I}

\Modulkuerzel{72025}

\EnglischerTitel{Fundamentals of Design I}

\SWS{4}

\ECTS{6} 

\Sprache{Deutsch}

\Modulverantwortlicher{\StudiendekanInf}

\Moduldauer{1}

\Turnus{
\periodisch{\WiSe}
}

\Dozenten{
\LB{Roger Walk}
\LB{Roland Barth}
}
% Schwerpunkt (Bachelor)
% Kernfächer: Praktische Informatik, Angewandte Informatik, Technische Informatik, Theoretische Informatik
% Vertiefungsfach (Master)
% Projektfach (Master)
%    Ein Projektmodul darf ausschließlich Veranstaltungen der Form eines
%    Projektes, Laborpraktikums oder eines Seminars beinhalten und umfasst immer 16 LP
\Einordnung{
\MedInf{\Ba}{\Pflichtfach}{\MEI}
}

\VoraussetzungenInhaltlich{Keine}
\VoraussetzungenFormal{Keine}

\Lernziele{
Die Studierenden sind in der Lage grundlegende Elemente der Gestaltung praktisch einzusetzen.
Die gestalterischen Mittel der Typografie und Dramaturgie können konzeptionell
und praktisch angewandt werden. Das darüber hinaus vermittelte Wissen über Wahrnehmungstheorie und Medienwissenschaft
befähigt die Studierenden zur Beurteilung bestehender Systeme.}

\Inhalt{
\spiegelstrich{Grundelemente der Gestaltung: Punkt, Linie, Fläche. Raum, Zeit Farbe}
\spiegelstrich{Strukturelle Merkmale und formale Möglichkeiten der Grundelemente}
\spiegelstrich{Typographie: Funktion, Lesbarkeit, Ausdrucksmöglichkeiten}
\spiegelstrich{Dramaturgie: Sprache und Möglichkeiten des Mediums Bewegtbild (Film/Animation)}
\spiegelstrich{Wahrnehmungstheorie}
\spiegelstrich{Medienwissenschaft}
\spiegelstrich{Designgeschichte}
}

\Literatur{
\buch{Markus Dahm: Grundlagen der Mensch-Computer-Interaktion, Pearson, 2006}
\buch{Adrian Frutiger: Der Mensch und seine Zeichen, Fourier, 1978}
\buch{Armin Hofmann: Methodik der Form und Bildgestaltung, Niggli, 1965}
\buch{Herbert Kapitzki: Programmiertes Gestalten, Dieter Gitzel, 1980}
\buch{Cyrus Khazaeli: Systemisches Design, Rowolth, 2005}
\buch{Pina Lewandowski: Visuelles Gestalten mit dem Computer, Rowolth, 2002}
\buch{William Lidwell: Universal principles of Design, Rockport, 2003}
\buch{John Maeda: Creative Code, Birkhäuser, 2004}
\buch{John Maeda: Maeda \@ Media, Thames \& Hudson, 2000}
\buch{John Maeda: Design by Numbers, Link: http://dbn.media.mit.edu, 1999}
\buch{Jef Raskin: Das intelligente Interface, Addison-Wesley, 2001}
}

\Lehrformen{
\Vlg{Grundlagen der Gestaltung I}{Roger Walk}
\Ubg{Grundlagen der Gestaltung I}{Roger Walk}
}

\Arbeitsaufwand{
\Praesenzzeit{120}    
\VorNachbereitung{60}
\Summe{180}
}

\Leistungsnachweis{Der Leistungsnachweis erfolgt durch eine Dokumentation der eigenen Implementierungen und Entwürfe als Belegarbeit.}

\Notenbildung{Das Modul ist unbenotet. Bei ausreichenden Leistungen wird ein Leistungsnachweis (Schein) vergeben.}

\Grundlagen{
Modul Grundlagen der Gestaltung II.
}

