\svnid{$Id$}

%\input{header}
%\begin{document}
%\input{macros}


%\input{header.tex}

\Modultitel{Praktikum - Halbleitertechnologie}

\Modulkuerzel{70446}

\EnglischerTitel{Semiconductor Technology Lab}

\SWS{
%An dieser Stelle werden die Präsenzzeiten angegeben (V/Ü)
%4+2 (V/Ü)
4
}

\ECTS{
%Anzahl der Leistungspunkte
5
}


\Sprache{
%Unterrichtssprache
deutsch
}

\Moduldauer{1} %Hier steht ob das Modl ein oder zwei Semester lang ist.

\Turnus{
%\sporadisch{\WiSe 2010}
\periodisch{\Se}
%\periodisch{\SoSe}
}

\Dozenten{
  %Ein Hochschullehrer, Honorarprofessor, Privatdozent, Gastprofessor oder ein
  %Lehrbeauftragter gemäß §44 LHG Baden-Württemberg. Hier stehen alle Hochschullehrer, 
  %die am Modul mitwirken.

\LB{Dr. Wolfgang Ebert}
}


\Modulverantwortlicher{
%Ein Dozent mit Prüfungsberechtigung, der ein Modul einrichtet und pflegt.
 \LB{Dr. Wolfgang Ebert}
}

\Einordnung{
  \ET{\Ma}{\Wahlpflichtmodul}{\Ingwi}
 
  \ET{\Ma}{\Pflichtmodul}{\Mikro} 
 \ET{\Ma}{\Wahlmodul}{\AUT}
  \Comm{\Ma}{\EmpfohlenesWahlmodul}{}
}


\VoraussetzungenInhaltlich{
	Vorlesung "MOS-Halbleitertechnik"\ oder "Modern Semiconductor Devices"

}

\VoraussetzungenFormal{
%keine
}

\Lernziele{
%Hier soll ein Überbick über die fachlichen Inhalte gegeben werden. 
Nach der erfolgreichen Absolvierung des Halbleitertechnologie-Praktikums sind die 
Teilnehmer praktisch befähigt, unter Reinraumbedingungen zu experimentieren und dabei 
komplexe Anlagen der Halbleitertechnologie zu bedienen. Sie modifizieren 
Halbleiteroberflächen durch thermische Abscheidung von Metallen wie Gold und Aluminium, 
um anschließend unter Verwendung dieser dünnen Metallschichten mittels Kontaklithografie 
mikrostrukturierte Metallkontakte zu generieren. Die Teilnehmer analysieren 
Teststrukturen, Dioden und Transistoren unter Anwendung elementarer elektrischer 
Charakterisierungsverfahren und evaluieren,  wie sich äußere (Geometrie) - und innere  
Einflußgrößen (Dotierung, intrinsische Widerstände, Idealitätsfaktor) auf die Kennlinien 
der Bauelemente auswirken.
}
		

\Inhalt{
	%Hier soll ein Überbick über die fachlichen Inhalte gegeben werden.
	Ziel das Praktikums ist es, voll funktionstüchtige Feldeffekttransistoren (GaAs-
	MESFET`s) herzustellen und elektrisch zu charakterisieren.
	Das Praktikum findet in einem eigens dafür ausgestattetem Reinraum statt und
	vermittelt deshalb auch wichtige Erkenntnisse über die Tätigkeiten, technischen
	Anlagen sowie Verhaltensweisen in Reinräumen.
	
	Schwerpunkte des Praktikums sind:
	\spiegelstrich{ Abscheidung von Metallen im Vakuum}
	\spiegelstrich{ Strukturierung der Bauelemente mittels optischen Lithographieverfahren im Mikrometerbereich}
	\spiegelstrich{ Metallätzverfahren}
	\spiegelstrich{ Herstellung von sperrfreien und sperrenden Kontakten}
	\spiegelstrich{ Elektrische Charakterisierung der Bauelemente}

}
 
\Literatur{ 
%Für das Modul empfohlene Bücher oder Aufsätze.
\buch{ H. Beneking: "Halbleitertechnologie", Teubner, Stuttgart}
\buch{ D. Widmann, H. Mader, H. Friedrich: "Technologie hochintegrierter
Schaltungen", Halbleiterelektronik Bd. 19, Springer}
\buch{ W. Kellner, H. Kniekamp: "GaAs-Feldeffekttransistoren", Springer}


}


\Lehrformen{
\Pra{``Halbleitertechnologie'', 4~SWS}{}
}

\Arbeitsaufwand{%Hier soll einheitlich die Präsenzzeit angegeben werden. Dabei wird kein
               %Unterschied zwischen Sommersemester und Wintersemester gemacht. 

	   %\PraAufw{35}{115} \\
               %\Summe{150}
	      \Praesenzzeit{35}
	      \VorNachbereitung{115}\\
	      %\Selbststudium{}
	      %\Gruppenarbeit{}
	      \Summe{150}
}

\Leistungsnachweis{
Testat bei Erfüllung folgender Kriterien: erfolgreiche Teilnahme am
Praktikum, aktive Diskussion während des Kolloquiums, Abgabe und evtl.
Korrektur des Versuchsprotokolls

}

\Notenbildung{
Unbenotete Veranstaltung (Leistungsnachweis)
}

\Grundlagen{
Masterarbeit im Bereich Mikrofabrikation
 }




% \Ilias{
% Optional.
% }

%\input{ausgabe}


%\end{document}
