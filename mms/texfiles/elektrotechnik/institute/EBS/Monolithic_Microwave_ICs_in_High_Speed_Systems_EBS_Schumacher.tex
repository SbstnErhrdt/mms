\svnid{$Id$}

%\input{header}
%\begin{document}
%\input{macros}


%\input{header.tex}

\Modultitel{Monolithic Microwave ICs in High-Speed Systems}

\Modulkuerzel{70457}

\DeutscherTitel{-}

\SWS{
%An dieser Stelle werden die Präsenzzeiten angegeben (V/Ü)
%4+2 (V/Ü)
4
}

\ECTS{
%Anzahl der Leistungspunkte
6
}


\Sprache{
%Unterrichtssprache
englisch
}

\Moduldauer{1} %Hier steht ob das Modl ein oder zwei Semester lang ist.

\Turnus{
%\sporadisch{\WiSe 2010}
\periodisch{\SoSe}
%\periodisch{\SoSe}
}

\Dozenten{
  %Ein Hochschullehrer, Honorarprofessor, Privatdozent, Gastprofessor oder ein
  %Lehrbeauftragter gemäß §44 LHG Baden-Württemberg. Hier stehen alle Hochschullehrer, 
  %die am Modul mitwirken.

\Prof{Dr.-Ing. Hermann Schumacher}
\Prof{Dr.-Ing. Heinrich Dämbkes}
  \LB{Dr. Vaclav Valenta}
}


\Modulverantwortlicher{
%Ein Dozent mit Prüfungsberechtigung, der ein Modul einrichtet und pflegt.
 \Prof{Dr.-Ing. Hermann Schumacher}
}

\Einordnung{
  \ET{\Ma}{\Wahlpflichtmodul}{\Ingwi}
 
  \ET{\Ma}{\Wahlpflichtmodul}{\Mikro}\\ 
  \ET{\Ma}{\Wahlmodul}{\AUT}\\ 
  \Comm{\Ma}{\Pflichtmodul}{\Mikro}

}


\VoraussetzungenInhaltlich{
	Undergraduate knowledge of analog circuit design and RF engineering principles


}

\VoraussetzungenFormal{
%keine
}

\Lernziele{
%Hier soll ein Überbick über die fachlichen Inhalte gegeben werden. 
Students recognize the impact of system level requirements on circuit design approaches. 
They then discuss key building blocks of radio frequency subsystems and circuits, 
identifying important circuit design concepts for high-speed wireless systems. They 
then apply these concepts, generating and designing key circuit blocks in lab exercises. 
Students also recognize key properties of the industry-standard CAD tool ADS, and 
practice its use in their designs.
}
		

\Inhalt{
	%Hier soll ein Überbick über die fachlichen Inhalte gegeben werden.
	Building blocks of a satellite receiving system and their characteristic
	specifications.

	Microwave circuit components:
	\spiegelstrich{Passive components and their equivalent circuits}
	\spiegelstrich{CAD models for III-V FETs, MOSFETs and hetero-bipolar transistors}
	\spiegelstrich{Transmission lines}
	\spiegelstrich{Micro-electro-mechanical structures for RF applications}
	\spiegelstrich{Realization of reactances and impedance converters using transmission lines}
	Circuit realization of building blocks:
	\spiegelstrich{Low-noise amplifiers}
	\spiegelstrich{Active and passive mixers}
	\spiegelstrich{Microwave oscillators}
	\spiegelstrich{Wide-band intermediate frequency amplifiers}
	\spiegelstrich{Multi-functional MMICs}
	\spiegelstrich{Packaging aspects and high-frecquency microsystems}
	

}
 
\Literatur{ 
%Für das Modul empfohlene Bücher oder Aufsätze.

\buch{D.M. Pozar: Microwave Engineering , Addison-Wesley, 1990}
\buch{ G.D. Vendelin, A.M. Pavio, U.L.Rohde: Microwave Circuit Design Using Linear
and Nonlinear Techniques , Wiley, 1990}
\buch{ R.E. Collin: Foundation for Microwave Engineering , McGraw-Hill, 1992}
\buch{ R.S. Elliott: An Introduction to Guided Waves and Microwave Circuits , Prentice-
Hall, 1993}
\buch{ G.Matthaei, L. Young \& E. Jones: Microwave Filters, Impedance-Matching
Networks \& Coupling Structures , Artech House, 1980}
\buch{ C.G. Montgomery, R.H. Dicke and E.M. Purcell: Principles of Microwave Circuits
(reprint of Radiation Laboratory volume 8 ) , IEEE Press, 1987}
\buch{ F.E.Gardiol: Introduction to Microwaves , Artech House, 1984}
\buch{ Bahl and P. Bhartia: Microwave Solid-State Circuit Design , Wiley, 1988}
\buch{ S. Prasad, H. Schumacher, A. Gopinath, High Speed Electronics and Optoelectronics 
(Chapter 5)}

}


\Lehrformen{

\Vlg{``Monolithic Microwave ICs in High-Speed Systems'', 2~SWS}{}
\Ubg{``Monolithic Microwave ICs in High-Speed Systems'', 1~SWS}{}
\Lab{``Simulation Lab'', 1~SWS}{}
\Tut{: Online Tutorials on microwave CAD techniques}{}  
}

\Arbeitsaufwand{%Hier soll einheitlich die Präsenzzeit angegeben werden. Dabei wird kein
               %Unterschied zwischen Sommersemester und Wintersemester gemacht. 

               %\VlgAufw{32}{20}    % 15 * SWS
               %\UebAufw{16}{20}
	    %\\{\bf Projekt:} Präsenzzeit: 12 h, Vor- und Nachbearbeitung: 20 h
	    %\Exam{60}\\
             %  \Summe{180}
	    \Praesenzzeit{60}
	    \VorNachbereitung{120}\\
	    %\Gruppenarbeit{}
	    \Summe{180}
}

\Leistungsnachweis{
Admission to exam requires participation in at least 6 out of 9 exercise sessions (in class and laboratory).\newline
Oral exam composed of an oral presentation on an MMIC journal paper and a
question and answer session (45 minutes)

}

\Notenbildung{
Grade of module is equal to grade of the oral exam
}

\Grundlagen{
Master thesis on microwave IC design
 }




% \Ilias{
% Optional.
% }

%\input{ausgabe}


%\end{document}
