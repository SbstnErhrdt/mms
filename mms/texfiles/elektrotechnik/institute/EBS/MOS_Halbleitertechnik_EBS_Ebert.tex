\svnid{$Id$}

%\input{header}
%\begin{document}
%\input{macros}


%\input{header.tex}

\Modultitel{MOS Halbleitertechnik}

\Modulkuerzel{70449}

\EnglischerTitel{-}

\SWS{
%An dieser Stelle werden die Präsenzzeiten angegeben (V/Ü)
%4+2 (V/Ü)
4
}

\ECTS{
%Anzahl der Leistungspunkte
6
}


\Sprache{
%Unterrichtssprache
deutsch
}

\Moduldauer{1} %Hier steht ob das Modl ein oder zwei Semester lang ist.

\Turnus{
\periodisch{\WiSe}
%\periodisch{\SoSe}
}

\Dozenten{
  %Ein Hochschullehrer, Honorarprofessor, Privatdozent, Gastprofessor oder ein
  %Lehrbeauftragter gemäß §44 LHG Baden-Württemberg. Hier stehen alle Hochschullehrer, 
  %die am Modul mitwirken.

\LB{Dr. Wolfgang Ebert}
}


\Modulverantwortlicher{
%Ein Dozent mit Prüfungsberechtigung, der ein Modul einrichtet und pflegt.
 \LB{Dr. Wolfgang Ebert}
}

\Einordnung{
  \ET{\Ma}{\Wahlpflichtmodul}{\Ingwi}
 
  \ET{\Ma}{\Pflichtmodul}{\Mikro}\\
  \ET{\Ma}{\Wahlmodul}{\AUT}\\
}


\VoraussetzungenInhaltlich{
	Grundlagen der Halbleiterbauelemente


}

\VoraussetzungenFormal{
%keine
}

\Lernziele{
%Hier soll ein Überbick über die fachlichen Inhalte gegeben werden. 
Die Ziele der Vorlesung MOS-Halbleitertechnologie bestehen darin, dass die Studenten  
ausführlich alle grundlegenden technologischen Prozesse benennen können, die für die 
Herstellung von MOS-Transistoren erforderlich sind. Die Studenten erkennen, wie 
MOS-Transistoren aufgebaut sind und beschreiben, auf welchen physikalischen Prinzipien 
deren Wirkungsweise beruht. Dadurch werden sie befähigt, Transistorkennlinien zu 
interpretieren und die Kontakt- und Halbleitereigenschaften zu charakterisieren. 
Sie differenzieren zwischen verschiedene Herstellungstechnologien (n-MOS, CMOS) und 
beschreiben grundlegende Anwendungen von MOS-Transistoren (Schalter, Inverter). 
Abschließend diskutieren die Studenten, welche Probleme die Miniaturisierung der 
Transistoren in technologischer- und physikalischer Hinsicht hervorruft erklären, 
wie diese Probleme gelöst werden können.
}
		

\Inhalt{
	%Hier soll ein Überbick über die fachlichen Inhalte gegeben werden.
	Einführung in die grundlegenden Prozesse der Si-Planartechnologie,
physikalische Grundlagen des MOSFET, MOSFET als Kleinsignalverstärker, Schalter und Inverter (n-MOS, CMOS)

}
 
\Literatur{ 
%Für das Modul empfohlene Bücher oder Aufsätze.

\buch{Sze, S.M.:Physics of Semiconductor Devices, John Wiley \& Sons Inc., New York, 1981}
\buch{Pierret, R.F.:Semiconductor Fundamentals, 
(Modular Series on Solid State Devices, Vol. 1 ), Addison-Wesley Reading, 1988}

}


\Lehrformen{

\Vlg{``MOS Halbleitertechnik'', 2~SWS}{}
\Ubg{``MOS Halbleitertechnik'', 2~SWS}{}  
}

\Arbeitsaufwand{%Hier soll einheitlich die Präsenzzeit angegeben werden. Dabei wird kein
               %Unterschied zwischen Sommersemester und Wintersemester gemacht. 

               %\VlgAufw{27}{50}    % 15 * SWS
               %\UebAufw{12}{41}
	       %\\{\bf Projekt:} Präsenzzeit: 12 h, Vor- und Nachbearbeitung: 20 h
	       %\Exam{50}\\
		\Praesenzzeit{60}
		\VorNachbereitung{70}
		\Selbststudium{50}\\
		%\Gruppenarbeit{}
		\Summe{180}
}

\Leistungsnachweis{
in der Regel mündliche Prüfung, sonst schriftliche Prüfung von 120 Minuten Dauer.

}

\Notenbildung{
Modulnote ist identisch mit Prüfungsnote
}

\Grundlagen{
keine Angaben
 }




% \Ilias{
% Optional.
% }

%\input{ausgabe}


%\end{document}
