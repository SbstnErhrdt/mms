\svnid{$Id$}

%\input{header}
%\begin{document}
%\input{macros}


%\input{header.tex}

\Modultitel{Lab - Semiconductor Technology}

\Modulkuerzel{70446}

\DeutscherTitel{-}

\SWS{
%An dieser Stelle werden die Präsenzzeiten angegeben (V/Ü)
%4+2 (V/Ü)
4
}

\ECTS{
%Anzahl der Leistungspunkte
5
}


\Sprache{
%Unterrichtssprache
English
}

\Moduldauer{1} %Hier steht ob das Modl ein oder zwei Semester lang ist.

\Turnus{
\periodisch{\Se}
}

\Dozenten{
  %Ein Hochschullehrer, Honorarprofessor, Privatdozent, Gastprofessor oder ein
  %Lehrbeauftragter gemäß §44 LHG Baden-Württemberg. Hier stehen alle Hochschullehrer, 
  %die am Modul mitwirken.
  \LB{Dr. Wolfgang Ebert}
 \JunProf{Dr.-Ing. Steffen Strehle}
}


\Modulverantwortlicher{
%Ein Dozent mit Prüfungsberechtigung, der ein Modul einrichtet und pflegt.
 \JunProf{Dr.-Ing. Steffen Strehle}

}

\Einordnung{
  \ET{\Ma}{\Wahlpflichtmodul}{\Ingwi}
    \Comm{\Ma}{\Wahlpraktikum}{\Mikro}
  }

\VoraussetzungenInhaltlich{
Lecture Semiconductor Technology
}


\VoraussetzungenFormal{
%keine
}

\Lernziele{
%Hier soll ein Überbick über die fachlichen Inhalte gegeben werden. 
Students discover how to work under clean-room conditions, recognizing how different 
technology steps may be combined to produce  electron devices. Furthermore, they 
practice to operate complex semiconductor-technology-equipment. Participants modify 
the surface of semiconductor wafers employing thermal evaporation of different metals 
like Aluminum and Gold. They create microstructured contacts by photo-lithography. 
Test structures, diodes, and transistors are evaluated using fundamental  
current-voltage and capacitance-voltage measurements. The students evaluate the 
influence of scaling parameters (geometry, size) on the electrical behaviour of the 
devices (output- and transfer-characteristics).
}
		

\Inhalt{
	%Hier soll ein Überbick über die fachlichen Inhalte gegeben werden.
Aim of this lab course is the fabrication of field-effect-transistors
(GaAs-MESFET's) and their electrical characterization.
The lab course takes place in a cleanroom facility specifically equipped
for education.\\

Main focuses this lab course are:

    \spiegelstrich{deposition of metals by evaporation}
    \spiegelstrich{patterning of contacts by optical lithography}
    \spiegelstrich{patterning of contacts by wet etching}
    \spiegelstrich{manufacturing of ohmic- and Schottky contacts}
    \spiegelstrich{electrical characterization of the fabricated devices}

}
 
\Literatur{ 
%Für das Modul empfohlene Bücher oder Aufsätze.
	\buch{Jackson, K. A. / Schröter, W. (Hrsg.) Handbook of Semiconductor Technology, 2000 ISBN-13: 978-3-527-29970-6 - Wiley-VCH, Weinheim} 

	\buch{Pierret, R.F.: Field Effect Devices, (Modular Series on Solid State Devices, Vol. 4), Addison- Wesley Reading, 1983}

	\buch{Zambuto, M.: Semiconductor Devices, McGraw Hill New York, 1989}

	\buch{C. Y. Chang, F. Kai: GaAs High-Speed Devices: Physics, Technology, and Circuit Applications Wiley, 1994}
}


\Lehrformen{
\Lab{``Semiconductor Technology'', 4~SWS}{}
}

\Arbeitsaufwand{%Hier soll einheitlich die Präsenzzeit angegeben werden. Dabei wird kein
               %Unterschied zwischen Sommersemester und Wintersemester gemacht. 
    %\VorNachbereitung{122}
    %\Praesenzzeit{28}
    %\Summe{150}
    \Praesenzzeit{28}
    \VorNachbereitung{122}\\
    %\Selbststudium{}
    %\Gruppenarbeit{}
    \Summe{150}
}

\Leistungsnachweis{ 
Successful completion requires fulfillment of the following criteria: successful participation 
in the lab, active participation during the colloquium, submission and correction 
(where applicable) of the experiment notes.
}

\Notenbildung{
None
}

\Grundlagen{
-
}
  
% \Ilias{
% Optional.
% }

%\input{ausgabe}


%\end{document}
