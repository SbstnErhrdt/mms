\svnid{$Id$}

%\input{header}
%\begin{document}
%\input{macros}


%\input{header.tex}

\Modultitel{Cultural Crossroads}

\Modulkuerzel{71729}

\DeutscherTitel{-}

\SWS{
%An dieser Stelle werden die Präsenzzeiten angegeben (V/Ü)
%4+2 (V/Ü)
2
}

\ECTS{
%Anzahl der Leistungspunkte
2}

\Sprache{
%Unterrichtssprache
Englisch
}

\Moduldauer{1} %Hier steht ob das Modl ein oder zwei Semester lang ist.

\Turnus{
\periodisch{\Se}
}

\Dozenten{
  %Ein Hochschullehrer, Honorarprofessor, Privatdozent, Gastprofessor oder ein
  %Lehrbeauftragter gemäß §44 LHG Baden-Württemberg. Hier stehen alle Hochschullehrer, 
  %die am Modul mitwirken.

  \LB{Dr.rer.nat. Katrin Reimer}
  
}


\Modulverantwortlicher{
%Ein Dozent mit Prüfungsberechtigung, der ein Modul einrichtet und pflegt.
  \Prof{Dr.-Ing. Hermann Schumacher}
}

\Einordnung{
  \ET{\Ma}{\Wahlpflichtmodul}{\Ingwi}
  \Comm{\Ma}{\NTechWahlmodul}{}
  
} 

\VoraussetzungenInhaltlich{
	\spiegelstrich {Basic skills in using presentation software (PowerPoint, Open Office Impress or similar)}
	\spiegelstrich {Good knowledge of English}
}

\VoraussetzungenFormal{
keine
}
\Lernziele{
	%Hier soll ein Überbick über die fachlichen Inhalte gegeben werden.
Students acquire basic presentation skills. The student selects relevant 
topics about his or her home country, recognizes aspects of general 
interest and researches on them, and prepares an oral   presentation. 
Peculiarities of the own home culture will be identified, desribed and 
reflected.
Students will practice discussing issues and defending their own 
statements. 
Students acquire intercultural competence by relating to their peers 
of other countries in a multicultural group.	
}
\Inhalt{
	%Hier soll ein Überbick über die fachlichen Inhalte gegeben werden.
	\spiegelstrich {Preparation of a presentation using guidelines}
	\spiegelstrich {Oral presentation} 
	\spiegelstrich {Group discussion on content}
	
}


\Literatur{ 

%Für das Modul empfohlene Bücher oder Aufsätze.
	
}

\Lehrformen{
\Seminar{``Cultural Crossroads'', 2~SWS (V)}{}

}

\Arbeitsaufwand{%Hier soll einheitlich die Präsenzzeit angegeben werden. Dabei wird kein
               %Unterschied zwischen Sommersemester und Wintersemester gemacht. 
               \Praesenzzeit{26}    
               \VorNachbereitung{34}\\
               \Summe{60}
}

\Leistungsnachweis{
Participation in seminar compulsory (students may miss a maximum of two dates) 
seminar presentation (30 min)
}

\Notenbildung{
Not graded (pass/fail)
}

\Grundlagen{

}

% \Ilias{
% Optional.
% }

%\input{ausgabe}


%\end{document}
