\svnid{$Id$}

%\input{header}
%\begin{document}
%\input{macros}


%\input{header.tex}

\Modultitel{Technical Presentation Skills}

\Modulkuerzel{71730}

\DeutscherTitel{-}

\SWS{
%An dieser Stelle werden die Präsenzzeiten angegeben (V/Ü)
%4+2 (V/Ü)
2
}

\ECTS{
%Anzahl der Leistungspunkte
2
}


\Sprache{
%Unterrichtssprache
englisch
}

\Moduldauer{1} %Hier steht ob das Modl ein oder zwei Semester lang ist.

\Turnus{
%\sporadisch{\WiSe 2010}
\periodisch{\WiSe}
%\periodisch{\SoSe}
}

\Dozenten{
  %Ein Hochschullehrer, Honorarprofessor, Privatdozent, Gastprofessor oder ein
  %Lehrbeauftragter gemäß §44 LHG Baden-Württemberg. Hier stehen alle Hochschullehrer, 
  %die am Modul mitwirken.

\Prof{Dr.-Ing. Hermann Schumacher}
}


\Modulverantwortlicher{
%Ein Dozent mit Prüfungsberechtigung, der ein Modul einrichtet und pflegt.
\Prof{Dr.-Ing. Hermann Schumacher}

}

\Einordnung{
  \Comm{\Ma}{\NTechWahlmodul}{}
}


\VoraussetzungenInhaltlich{
Basic knowledge of presentation software (e.g. PowerPoint, LibreOffice Impress)
Bachelor-level engineering background.

}

\VoraussetzungenFormal{
%keine
}

\Lernziele{
%Hier soll ein Überbick über die fachlichen Inhalte gegeben werden. 
Students recognize proven techniques for technical oral presentations supported by visual 
aids. In the preparation of their presentation, they distinguish different target audiences 
and devise their presentation strategy accordingly. They differentiate between different 
forms of presentation (oral, written, web-based) and develop suitable communication 
strategies. Students create an oral presentation on a topic of their choice within an 
annually changing topical framework, defend their ideas in front of their peers, and 
summarize their presentation in a two-page written report.
}
		

\Inhalt{
	%Hier soll ein Überbick über die fachlichen Inhalte gegeben werden.
	\spiegelstrich{ Presentation quality criteria}
	\spiegelstrich{ Researching a subject}
	\spiegelstrich{ Structuring oral presentations}
	\spiegelstrich{ Visual aids preparation}
	\spiegelstrich{ Multimedia techniques}
	\spiegelstrich{ Public speaking}
	\spiegelstrich{ Handling questions and critique}
	\spiegelstrich{ Written presentations:}
	\begin{enumerate}
		\item  Research reports
		\item Journal arcticles
		\item Theses
	\end{enumerate}
	\spiegelstrich{ Presenting technical matters on the web}
	\spiegelstrich{ Seminar trial presentations}




}
 
\Literatur{ 
%Für das Modul empfohlene Bücher oder Aufsätze.
keine Angaben
}


\Lehrformen{
\Vlg{``Technical Presentation Skills'', 1~SWS}{}
\Sem{``Technical Presentation Skills'', 1~SWS}{}  
}

\Arbeitsaufwand{%Hier soll einheitlich die Präsenzzeit angegeben werden. Dabei wird kein
               %Unterschied zwischen Sommersemester und Wintersemester gemacht. 

	    \Praesenzzeit{30}
	    \VorNachbereitung{30}\\
	    \Summe{60}
}

\Leistungsnachweis{

Participation in lectures and seminars (student can miss a maximum of three dates)
Public seminar presentation
Two-page written abstract
}
\Notenbildung{
non-graded
}

\Grundlagen{


}
  
% \Ilias{
% Optional.
% }

%\input{ausgabe}


%\end{document}
