\svnid{$Id$}

%\input{header}
%\begin{document}
%\input{macros}


%\input{header.tex}

\Modultitel{Grundlagen der Halbleiter-Bauelemente}

\Modulkuerzel{70383}

\EnglischerTitel{-}

\SWS{
%An dieser Stelle werden die Präsenzzeiten angegeben (V/Ü)
%4+2 (V/Ü)
6
}

\ECTS{
%Anzahl der Leistungspunkte
7
}


\Sprache{
%Unterrichtssprache
deutsch
}

\Moduldauer{1} %Hier steht ob das Modl ein oder zwei Semester lang ist.

\Turnus{
%\sporadisch{\WiSe 2010}
\periodisch{\WiSe}
%\periodisch{\SoSe}
}

\Dozenten{
  %Ein Hochschullehrer, Honorarprofessor, Privatdozent, Gastprofessor oder ein
  %Lehrbeauftragter gemäß §44 LHG Baden-Württemberg. Hier stehen alle Hochschullehrer, 
  %die am Modul mitwirken.

\JunProf{Dr.-Ing. Steffen Strehle}
}


\Modulverantwortlicher{
%Ein Dozent mit Prüfungsberechtigung, der ein Modul einrichtet und pflegt.
 \JunProf{Dr.-Ing. Steffen Strehle}
}

\Einordnung{
  \ET{\Ba}{\Pflichtmodul}{}
}


\VoraussetzungenInhaltlich{
	Inhalte des Physikunterrichts der gymnasialen Sekundarstufe II
}

\VoraussetzungenFormal{
%keine
}

\Lernziele{
%Hier soll ein Überbick über die fachlichen Inhalte gegeben werden. 
Die Studierenden sind in der Lage die Grundkonzepte der klassischen und 
quantenmechanischen Festkörperphysik zu beschreiben und bzgl. ihrer 
Anwendungsgebiete zu differenzieren. Aufbauend auf diesen Kompetenzen können 
sie Aufgaben zur Materiewellenlänge, zur Fermi-Dirac-Statistik, zum Tunneleffekt, 
zur effektiven Masse und zum Verhalten von Wellenfunktionen in Potentialkästen 
bzw. periodischen Potentialen unter Anwendung der eindimensionalen, stationären 
Schrödingergleichung lösen. Die Studierenden können die Ergebnisse erklären und 
darauf aufbauend Banddiagramme verschiedener Halbleiterwerkstoffe im k-Raum 
evaluieren. Sie sind in der Lage Einflüsse von der Zusammensetzung, 
Kristallorientierung, Dotierung und Temperatur auf die halbleitenden 
Eigenschaften generalisiert mathematisch zu beschreiben und hierauf basierend 
Lösungen auf konkrete Fragestellungen ableiten zu können. Dies umfasst auch die 
Evaluierung des stationären und dynamischen Verhaltens von Halbleitern unter 
Beachtung von Ladungsträgerinjektion, Rekombination, Generation und elektrischen 
Feldern. Auf diesen Kompetenzen aufbauend sind die Studierenden befähigt pn-, 
pin- sowie Schottky-Übergänge, MOS-Kondensatoren sowie den Bipolar- und 
Feldeffekt-Transistor zu beschreiben, die Funktionsprinzipen differenziert 
zu diskutieren, mathematische und funktionelle Zusammenhänge abzuleiten und 
Vorhersagen über das elektrische Verhalten und den jeweiligen Einsatzbereich zu 
treffen. Darüber hinaus können sie auch Ihnen bisher unbekannte Bauelemente 
kategorisieren und bzgl. anwendungsorientierter Fragestellungen beurteilen. 
}
		

\Inhalt{
	%Hier soll ein Überbick über die fachlichen Inhalte gegeben werden.
Die Vorlesung baut auf den Grundlagen der klassischen Atomphysik auf und erweitert 
diese um die fundamentalen Prinzipien der Quantenmechanik. Ausgehend von einfachen 
Wellengleichungen werden die Energiebanddiagramme im k-Raum, die 
Fermi-Dirac-Statistik, die effektive Masse und der Tunneleffekt systematisch 
abgeleitet. Diese Erkenntnisse bilden die Basis zum Verständnis des physikalischen 
Verhaltens von Elektronen und Löchern sowohl in intrinsischen als auch in 
dotierten Halbleiterkristallen. Darauf aufbauend werden verschiedene Halbleiterübergänge, 
als eine wesentliche Grundlage zum Verständnis zahlreicher Halbleiterbauelemente, 
besprochen und mathematisch und funktionell charakterisiert. Den abschließenden Teil 
der Vorlesung bildet die Analyse wichtigster Grundstrukturen, wie z.B. pn-Diode, 
Schottky-Kontakte, MOS-Kondensator und Bipolartransistoren, die einen Schlüssel 
zum Verständnis komplexerer Halbleitersysteme darstellen.
}
 
\Literatur{ 
%Für das Modul empfohlene Bücher oder Aufsätze.
\buch{Thuselt, R.: Physik der Halbleiterbauelemente, 2. Aufl., Springer-Verlag Berlin Heidelberg, 2011}
\buch{Grundmann, M.: The Physics of Semiconductors, 2nd Ed., Springer-Verlag Berlin Heidelberg, 2010}
\buch{Müller, R.: Grundlagen der Halbleiterelektronik, (Reihe: Halbleiter - Elektronik,
Bd. 1 ), Springer-Verlag, Berlin, 1987}
\buch{Pierret, R.F.: Semiconductor Fundamentals, (Modular Series on Solid State
Devices, Vol. 1), Addison-Wesley Reading, 1988}
}


\Lehrformen{

\Vlg{``Grundlagen der Halbleiter-Bauelemente'', 3~SWS}{}
\Ubg{``Grundlagen der Halbleiter-Bauelemente'', 2~SWS}{}
\Tut{``Grundlagen der Halbleiter-Bauelemente'', 1~SWS}{}
}

\Arbeitsaufwand{%Hier soll einheitlich die Präsenzzeit angegeben werden. Dabei wird kein
               %Unterschied zwischen Sommersemester und Wintersemester gemacht. 

		%\VlgAufw{30}{60}    % 15 * SWS
		%\UebAufw{20}{40}
		%\Exam{60}\\
		%\Summe{210}
		\Praesenzzeit{50}
		\VorNachbereitung{100}
		\Selbststudium{60}\\
		%\Gruppenarbeit{}
		\Summe{210}
}

\Leistungsnachweis{
schriftliche Prüfung von 120 Minuten Dauer
}

\Notenbildung{
Die Modulnote entspricht dem Ergebnis der schriftlichen Prüfung. 
Durch Erbringung einer Studienleistung (Vortrag mit Übung und schriftlicher 
Ausarbeitung) kann ein Notenbonus auf die Modulprüfung bis zur nächst 
besseren Zwischenstufe von 0,3 bzw. 0,4 gewährt werden. Eine 
Notenverbesserung von 5,0 auf 4,0 ist nicht möglich.
}

\Grundlagen{
Veranstaltungen zu Halbleiterbauelementen in Master-Studiengängen
}




% \Ilias{
% Optional.
% }

%\input{ausgabe}


%\end{document}
