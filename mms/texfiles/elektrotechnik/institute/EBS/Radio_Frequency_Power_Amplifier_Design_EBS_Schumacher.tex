\svnid{$Id$}

%\input{header}
%\begin{document}
%\input{macros}


%\input{header.tex}

\Modultitel{Radio Frequency Power Amplifier Design}

\Modulkuerzel{72075}

\DeutscherTitel{Hochfrequenz-Leitungsverstärker-Entwurf}

\SWS{
%An dieser Stelle werden die Präsenzzeiten angegeben (V/Ü)
%4+2 (V/Ü)
3
}

\ECTS{
%Anzahl der Leistungspunkte
4
}


\Sprache{
%Unterrichtssprache
Englisch
}

\Moduldauer{1} %Hier steht ob das Modl ein oder zwei Semester lang ist.

\Turnus{
%\sporadisch{\WiSe 2010}
\periodisch{\SoSe}
%\periodisch{\SoSe}
}

\Dozenten{
  %Ein Hochschullehrer, Honorarprofessor, Privatdozent, Gastprofessor oder ein
  %Lehrbeauftragter gemäß §44 LHG Baden-Württemberg. Hier stehen alle Hochschullehrer, 
  %die am Modul mitwirken.

    \LB{Dr.-Ing. Christoph Bromberger}
}

\Modulverantwortlicher{
%Ein Dozent mit Prüfungsberechtigung, der ein Modul einrichtet und pflegt.
  \Prof{Dr.-Ing. Hermann Schumacher}
  \LB{Dr.-Ing. Christoph Bromberger}
}

\Einordnung{
  \ET{\Ma}{\Wahlmodul}{\Ingwi}
\Comm{\Ma}{\Wahlmodul}{}
\IST{\Ma}{\Wahlmodul}{}
}

\VoraussetzungenInhaltlich{
Bachelor-Level Analog Design Skills
Some understanding of S-parameters is helpful
}


\Lernziele{
%Hier soll ein Überbick über die fachlichen Inhalte gegeben werden. 
After the first half of the lecture, the students identify the requirements from mobile 
communication systems for power amplifier design. Attendees recognize potential 
challenges in applying HF measurement equipment and employ techniques to circumvent them. 
Students differentiate between small- and large-signal operation and examine the 
respective strengths and limitations of time-domain and of S-parameter methods. They 
appraise for the real-world limitations to PA design, HF bandwidth and signal bandwidth.
After the second half, participants discriminate different efficiency enhancement 
concepts and apply load modulation amplifier design techniques.
}
		

\Inhalt{
	%Hier soll ein Überbick über die fachlichen Inhalte gegeben werden.
	\spiegelstrich {Data encoding, signal statistics and consequences for power
	amplifiers}
	\spiegelstrich {Measuring power devices} 
	\spiegelstrich {Small-signal vs. large-signal operation, LF vs. HF behavior}
	\spiegelstrich {Matching power devices using measurement and small-signal
	tools}
	\spiegelstrich {Some nasty shortcomings of nice theoretical approaches}
	\spiegelstrich {The Doherty amplifier and it’s design}
	\spiegelstrich {Outphasing, pre-distortion, feed-forward and switching mode
	Pas}
	\spiegelstrich {In the exercises, students engross in the concepts by applying
	them with the help of ADS to real-world designs.}
}

\Literatur{ 
%Für das Modul empfohlene Bücher oder Aufsätze.
	 
\buch {Steve Cripps, RF Power Amplifiers}
	
}

\Lehrformen{
\Projekt{``Radio Frequency Electronics'', Introductory lectures, 1 SWS (V)}{}
\Projekt{``Radio Frequency Electronics'' , design consulting sessions; guided design exercises; guided characterization exercises, 2 SWS (P)}{}
}

\Arbeitsaufwand{%Hier soll einheitlich die Präsenzzeit angegeben werden. Dabei wird kein
               %Unterschied zwischen Sommersemester und Wintersemester gemacht. 
	%\VlgAufw{20}{20}
	%\UebAufw{20}{20}
	%\PraAufw{20}{20}
	%\Exam{30}\\
	
  
         %      \Summe{150}
	\Praesenzzeit{39}
	\Selbststudium{25}
	\VorNachbereitung{26}
	\Selbststudium{30}
	\\
	%\Gruppenarbeit{}
	\Summe{120}
}

\Leistungsnachweis{
Usually written exam of 120 minutes, otherwise oral exam; pre-requisite for
participating in the exam: 60 per cent of the credits from the exercises }

\Notenbildung{
Module mark is equal to exam mark.
}

\Grundlagen{
Master thesis with radio frequency electronics content
}
  
% \Ilias{
% Optional.
% }

%\input{ausgabe}


%\end{document}
