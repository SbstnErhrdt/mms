\svnid{$Id$}

%\input{header}
%\begin{document}
%\input{macros}


%\input{header.tex}

\Modultitel{Project - Radio Frequency Electronics}

\Modulkuerzel{71563}

\DeutscherTitel{-}

\SWS{
%An dieser Stelle werden die Präsenzzeiten angegeben (V/Ü)
%4+2 (V/Ü)
3
}

\ECTS{
%Anzahl der Leistungspunkte
5
}


\Sprache{
%Unterrichtssprache
Englisch
}

\Moduldauer{1} %Hier steht ob das Modl ein oder zwei Semester lang ist.

\Turnus{
%\sporadisch{\WiSe 2010}
\periodisch{\SoSe}
%\periodisch{\SoSe}
}

\Dozenten{
  %Ein Hochschullehrer, Honorarprofessor, Privatdozent, Gastprofessor oder ein
  %Lehrbeauftragter gemäß §44 LHG Baden-Württemberg. Hier stehen alle Hochschullehrer, 
  %die am Modul mitwirken.

  \Prof{Dr.-Ing. Hermann Schumacher}
}

\Modulverantwortlicher{
%Ein Dozent mit Prüfungsberechtigung, der ein Modul einrichtet und pflegt.
  \Prof{Dr.-Ing. Hermann Schumacher}
  \LB{Dr.-Ing. Andreas Trasser}
}

\Einordnung{
  \ET{\Ma}{\Wahlpflichtmodul}{\Ingwi}
\Comm{\Ma}{\Wahlpraktikum}{}

}

\VoraussetzungenInhaltlich{
{Radio Frequency Engineering; recommended additionally: High Frequency
Microsystems or Monolithic Microwave ICs in High-Speed Systems}
}


\Lernziele{
%Hier soll ein Überbick über die fachlichen Inhalte gegeben werden. 
Students analyze an initially incomplete set of technical requirements. They discuss the 
requirements, identifying missing information and break down the project into individual 
tasks. Recognizing the necessary radio frequency electronics concepts to be employed, they 
sketch a solutionn path and design the necessary components using off-the-shelf components, 
appraising issues like availability and bill of materials. A prototype is finally fully 
developed, constructed and assessed. Along the way, common pitfalls of working in projects, 
and proven project management issues are reviewed. 
}
		

\Inhalt{
	%Hier soll ein Überbick über die fachlichen Inhalte gegeben werden.
This project will realize a different radio frequency subsystem each year. The
docents will describe a set of requirements, students will then set out to develop a
system concept, research suitable off-the-shelf components, perform the complete
design, and finally build and characterize a prototype. A written report describing
design decisions, all data relevant to replicate the prototype, and characterization
results will finalize the project.
Docents will act as design consultants; additionally, short lectures will introduce
important project management approaches such as Scrum, students will use
simple project management techniques during the design and implementation
phases.

}

\Literatur{ 
%Für das Modul empfohlene Bücher oder Aufsätze.
	 
\buch {M. Hoffmann, Hochfrequenztechnik -ein systemtheoretischer Zugang (in
German)}

\buch{Lecture notes for RF Engineering}
  
\buch {David Rutledge, The Electronics of Radio} 

\buch {S. Prasad/H. Schumacher/A. Gopinath, High-Speed Electronics and
Optoelectronics (Chapter 5)}
	
}

\Lehrformen{
\Projekt{``Radio Frequency Electronics'', Introductory lectures, 1 SWS (V)}{}
\Projekt{``Radio Frequency Electronics'' , design consulting sessions; guided design exercises; guided characterization exercises, 2 SWS (P)}{}
}

\Arbeitsaufwand{%Hier soll einheitlich die Präsenzzeit angegeben werden. Dabei wird kein
               %Unterschied zwischen Sommersemester und Wintersemester gemacht. 
	%\VlgAufw{20}{20}
	%\UebAufw{20}{20}
	%\PraAufw{20}{20}
	%\Exam{30}\\
	
  
         %      \Summe{150}
	\Praesenzzeit{60}
	\VorNachbereitung{60}
	\Selbststudium{30}\\
	%\Gruppenarbeit{}
	\Summe{150}
}

\Leistungsnachweis{
Continuous evaluation of progress during design consulting contact sessions;
written design report, oral presentation of individual work
contributions.
}

\Notenbildung{
ungraded course
}

\Grundlagen{
Master thesis with radio frequency electronics content
}
  
% \Ilias{
% Optional.
% }

%\input{ausgabe}


%\end{document}
