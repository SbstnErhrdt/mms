\svnid{$Id$}

%\input{header}
%\begin{document}
%\input{macros}


%\input{header.tex}

\Modultitel{Analoge Schaltungen}

\Modulkuerzel{70382}

\EnglischerTitel{-}

\SWS{
%An dieser Stelle werden die Präsenzzeiten angegeben (V/Ü)
%4+2 (V/Ü)
4
}

\ECTS{
%Anzahl der Leistungspunkte
4}

\Sprache{
%Unterrichtssprache
Deutsch
}

\Moduldauer{1} %Hier steht ob das Modl ein oder zwei Semester lang ist.

\Turnus{
%\sporadisch{\WiSe 2010}
\periodisch{\SoSe}
%\periodisch{\SoSe}
}

\Dozenten{
  %Ein Hochschullehrer, Honorarprofessor, Privatdozent, Gastprofessor oder ein
  %Lehrbeauftragter gemäß §44 LHG Baden-Württemberg. Hier stehen alle Hochschullehrer, 
  %die am Modul mitwirken.

  \LB{Dr.-Ing. Andreas Trasser}
  
}


\Modulverantwortlicher{
%Ein Dozent mit Prüfungsberechtigung, der ein Modul einrichtet und pflegt.
  \LB{Dr.-Ing. Andreas Trasser}
}

\Einordnung{
  \ET{\Ba}{\Pflichtmodul}{}\\
  \IST{\Ba}{\Wahlpflichtmodul}{}
} 

\VoraussetzungenInhaltlich{
	\spiegelstrich {Integral- und Differentialrechnung}
	\spiegelstrich {Inhalte der Vorlesung Grundlagen der Elektrotechnik I (insbes. Komplexe
Wechselstromrechnung, Analyse von Gleich- und Wechselstrom-Netzwerken,
gesteuerte Quellen)}
}

\VoraussetzungenFormal{
keine
}
\Lernziele{
	%Hier soll ein Überbick über die fachlichen Inhalte gegeben werden.
Die Studenten können die grundlegenden Eigenschaften von Bipolar- , Feldeffekttransistoren 
und Operationsverstärkern beschreiben und geeignete Bauelemente auswählen. Sie können komplexe 
analoge Schaltungen in bekannte Teilschaltungen partitionieren. Damit sind sie in der Lage, 
deren Funktion zu erkennen und ihr Verhalten zu  berechnen. Die Studenten können den Einfluss 
störender Größen wie Temperaturveränderungen, Betriebsspannungsschwankungen und 
Herstellungstoleranzen  quantifizieren und die Schaltung zur Stabilisierung modifizieren. 
Analoge Rechenschaltungen und Filterschaltungen können unter Verwendung von 
Operationsverstärkern  entworfen werden. Sie können die Unterschiede idealer und realer 
Operationsverstärker beurteilen und den Einfluss auf die Gesamtschaltung beschreiben.
}
\Inhalt{
	%Hier soll ein Überbick über die fachlichen Inhalte gegeben werden.
	\spiegelstrich {Lineare und nichtlineare Modellbildung aktiver Bauelemente}
	\spiegelstrich {Grundschaltungen aktiver Bauelemente} 
	\spiegelstrich {Erweiterte Grundschaltungen (z.B. Darlington, Kaskode, Differenzverstärker)}
	\spiegelstrich {Arbeitspunkt-Stabilisierung}
	\spiegelstrich {Elektronische Strom- und Spannungsquellen}
	\spiegelstrich {Grundlagen des Operationsverstärkers (OPV)}
	\spiegelstrich {Schaltungen mit OPV}
	% 	\spiegelstrich Korrektheit von Programmen (Hoare-Kalkül).
}


\Literatur{ 

%Für das Modul empfohlene Bücher oder Aufsätze.
	\buch{Tietze, U.; Schenk, Ch.: Halbleiterschaltungstechnik. 11. Auflage, Springer
Verlag, 1999
} 
	\buch{Horowitz, P, Hill, W., The Art of Electronics; Cambridge University Press} 
}

\Lehrformen{
\Vlg{``Analoge Schaltungen'', 2 SWS (V)}{}
\Ubg{``Analoge Schaltungen'', 1,5 SWS (Ü)}{}
\Lab{``Analoge Schaltungen'', 0,5 SWS (P)}{}
}

\Arbeitsaufwand{%Hier soll einheitlich die Präsenzzeit angegeben werden. Dabei wird kein
               %Unterschied zwischen Sommersemester und Wintersemester gemacht. 
               %\Praesenzzeit{56}    % 15 * SWS
               %\VorNachbereitung{94}
               %\Summe{120}
		\Praesenzzeit{48}
		\VorNachbereitung{72}
		%\Selbststudium{}
		%\Gruppenarbeit{}
		\Summe{120}
}

\Leistungsnachweis{
in der Regel schriftliche Prüfung von 120 Minuten Dauer, ansonsten mündliche
Prüfung. 
}

\Notenbildung{
Die Modulnote entspricht dem Ergebnis der schriftlichen Prüfung.
}

\Grundlagen{
Veranstaltungen des Master-Studiums mit starken analog-elektronischen Inhalten.
}

% \Ilias{
% Optional.
% }

%\input{ausgabe}


%\end{document}
