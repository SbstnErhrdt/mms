\svnid{$Id$}

%\input{header}
%\begin{document}
%\input{macros}


%\input{header.tex}

\Modultitel{Semiconductor Sensors}

\Modulkuerzel{70450}

\DeutscherTitel{Halbleitersensoren}

\SWS{
%An dieser Stelle werden die Präsenzzeiten angegeben (V/Ü)
%4+2 (V/Ü)
4
}

\ECTS{
%Anzahl der Leistungspunkte
5
}


\Sprache{
%Unterrichtssprache
Englisch
}

\Moduldauer{1} %Hier steht ob das Modl ein oder zwei Semester lang ist.

\Turnus{
%\sporadisch{\WiSe 2010}
\periodisch{\SoSe}
%\periodisch{\SoSe}
}

\Dozenten{
  %Ein Hochschullehrer, Honorarprofessor, Privatdozent, Gastprofessor oder ein
  %Lehrbeauftragter gemäß §44 LHG Baden-Württemberg. Hier stehen alle Hochschullehrer, 
  %die am Modul mitwirken.
  \LB{Dr. Alberto Pasquarelli}
}


\Modulverantwortlicher{
%Ein Dozent mit Prüfungsberechtigung, der ein Modul einrichtet und pflegt.
  \Prof{Dr.-Ing. Hermann Schumacher}
}

\Einordnung{
  \ET{\Ma}{\Wahlpflichtmodul}{\Ingwi}
  \ET{\Ma}{\Pflichtmodul}{\Mikro}
  \ET{\Ma}{\Wahlmodul}{\AUT}
  \Comm{\Ma}{\TechWahlmodul}{\Mikro}
  \ES{\Ma}{\Anwendungsfach}{Mixed Signal Systems}
  }

\VoraussetzungenInhaltlich{
Halbleiterbauelemente

}


\VoraussetzungenFormal{
%keine
}

\Lernziele{
%Hier soll ein Überbick über die fachlichen Inhalte gegeben werden. 
Learning objectives:

The advances in microelectronics and micro electro-mechanical systems (MEMS) 
have revolutionized the scenario of sensor technology. Thanks to new mateials 
and processes, traditional bulky, slow and expensive sensor systems could be 
replaced by miniaturized and integrated smart sensors based on semiconductors. 
With the help of semiconductor sensors various application areas have been 
developed. In everyday life we encounter them, for example in the form of 
navigation and control systems in vehicles or as microphones, accelerometers, 
compass and cameras in mobile phones and tablets. In addition to the automotive 
industry and the mobile communications, semiconductor sensors are used in many 
other areas, for example in health care to record the blood pressure or body 
temperature in real time. 

The students describe and classify principles of operation, technological 
implementations and application areas of different sensors. They recognize and 
discuss the various physical phenomena in semiconductors, which are used for 
the detection of physical quantities and their conversion to electrical signals. 
They know various semiconductor materials suitable for the production of 
sensors, analyze the peculiarities of each one, explain and predict their 
response under different conditions and can calculate sensor examples for 
different measurement needs. The students can design a semiconductor sensor 
choosing the right material among several semiconductors. They are able to 
analyze a measurement problem, compare appropriate sensing techniques and 
develop their own solution. Doing this they can properly dimension the sensor 
unit to meet the design specifications.
}
		

\Inhalt{
	%Hier soll ein Überbick über die fachlichen Inhalte gegeben werden.
	Semiconductor-based detection methods for:
	
	\spiegelstrich {radiation (ionizing and non-ionizing)}
	
	\spiegelstrich {magnetic fields} 
	
	\spiegelstrich {mechanical forces}
	
	\spiegelstrich {temperature}
	
	Basics on operational amplifiers
	Basics on MST (micro system technology)
	Basics on MEMS (micro electro-mechanical systems)
}
 
\Literatur{ 
%Für das Modul empfohlene Bücher oder Aufsätze.
	\buch{Sze, S.M.: Semiconductor sensors, John Wiley \& Sons, 1994} 
	
	\buch{Middelhoek S., Audet S.A., Silicon Sensors, Academic Press 1989}
	
	\buch{EPopovic R.S., Hall effect Devices, Adam Hilger, 1991} 
	
	\buch{Lecture Notes} 
}


\Lehrformen{
\Vlg{``Semiconductor Sensors'', 3~SWS (V)}{}
\Ubg{``Semiconductor Sensors'', 1~SWS (Ü)}{}  
}

\Arbeitsaufwand{%Hier soll einheitlich die Präsenzzeit angegeben werden. Dabei wird kein
               %Unterschied zwischen Sommersemester und Wintersemester gemacht. 
                 %\VlgAufw{42}{50}    % 15 * SWS
               %\UebAufw{14}{28}
	        %\Exam{16}\\
                %\Summe{120}
	        \Praesenzzeit{48}
		\VorNachbereitung{102}
		%\Gruppenarbeit{}
		\Summe{150}
}

\Leistungsnachweis{
Written exam of 120 minutes duration.
}

\Notenbildung{
Module mark is identical to exam mark.
}

\Grundlagen{
Master thesis in the area of semiconductor sensors.
}
  
% \Ilias{
% Optional.
% }

%\input{ausgabe}


%\end{document}
