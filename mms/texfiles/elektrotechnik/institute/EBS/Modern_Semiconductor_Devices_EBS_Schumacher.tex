\svnid{$Id$}

%\input{header}
%\begin{document}
%\input{macros}


%\input{header.tex}

\Modultitel{Modern Semiconductor Devices}

\Modulkuerzel{71228}

\DeutscherTitel{-}

\SWS{
%An dieser Stelle werden die Präsenzzeiten angegeben (V/Ü)
%4+2 (V/Ü)
3
}

\ECTS{
%Anzahl der Leistungspunkte
4}

\Sprache{
%Unterrichtssprache
English
}

\Moduldauer{1} %Hier steht ob das Modl ein oder zwei Semester lang ist.

\Turnus{
%\sporadisch{\WiSe 2010}
\periodisch{\SoSe}
%\periodisch{\SoSe}
}

\Dozenten{
  %Ein Hochschullehrer, Honorarprofessor, Privatdozent, Gastprofessor oder ein
  %Lehrbeauftragter gemäß §44 LHG Baden-Württemberg. Hier stehen alle Hochschullehrer, 
  %die am Modul mitwirken.

  \Prof{Dr.-Ing. Hermann Schumacher}
  
}


\Modulverantwortlicher{
%Ein Dozent mit Prüfungsberechtigung, der ein Modul einrichtet und pflegt.
  \Prof{Dr.-Ing. Hermann Schumacher}
}

\Einordnung{
  \ET{\Ma}{\Wahlpflichtmodul}{\Ingwi}
  \ET{\Ma}{\Wahlpflichtmodul}{\Mikro}\\
  \ET{\Ma}{\Wahlmodul}{\AUT}\\
  \Comm{\Ma}{\Pflichtmodul}{\Mikro}
} 

\VoraussetzungenInhaltlich{
	\spiegelstrich {Basic knowledge of solid-state physics: semiconductors; band structure in real and
in k-space; drift and diffusive transport}
}	

\VoraussetzungenFormal{
None
}
\Lernziele{
	%Hier soll ein Überbick über die fachlichen Inhalte gegeben werden.
Students recognize the importance of energy band diagrams for the analysis and appraisal of 
advanced electronic devices. They then discover how doping variations and heterostructures 
are used in current semiconductor devices to control the distribution and motion of free 
charge carriers. Relevant transistor structures are differentiated according to their charge 
control mechanism. Students then identify large signal and small signal equivalent circuits, 
and discuss how intrinsic physical mechanisms are reflected at the component and circuit 
level. They relate geometrical constraints of high speed transistor families to their 
economic importance, and briefly review important microfabrication techniques.
}
\Inhalt{
	%Hier soll ein Überbick über die fachlichen Inhalte gegeben werden.
	
	Semiconductor Fundamentals:

	\spiegelstrich{Energy band diagrams}
	\spiegelstrich{Doping}
	\spiegelstrich{MOS-, pn- and Schottky junctions}
	\spiegelstrich{Semiconductor Heterostructures}


	Electronic semiconductor devices:

	\spiegelstrich {MOSFET}
	\spiegelstrich {MESFET}
	\spiegelstrich {HEMT}
	\spiegelstrich {BJT}
	\spiegelstrich {HBT}

	Application aspects of semiconductor devices in RF/microwave communicationsystems:

	\spiegelstrich {Performance parameters}
	\spiegelstrich {System requirements}
	\spiegelstrich {Economical issues}

} 
	

\Literatur{ 

%Für das Modul empfohlene Bücher oder Aufsätze.
	\buch{Simon Sze, Physics of Semiconductor Devices
} 
	\buch{S. Prasad, H. Schumacher, A. Gopinath, High Speed Electronics and 
Optoelectronics (Chapter 1 and 2)
} 
	\buch{Full set of slides, video sequences on e-learning platform	
}
}
\Lehrformen{
\Vlg{``Modern Semiconductor Devices'', 2~SWS}{}
\Ubg{``Modern Semiconductor Devices'', 1~SWS}{}
\Lab{``Modern Semiconductor Devices'', 1x2 hours (one event)}{}
}

\Arbeitsaufwand{%Hier soll einheitlich die Präsenzzeit angegeben werden. Dabei wird kein
               %Unterschied zwischen Sommersemester und Wintersemester gemacht. 
               %\VlgAufw{25}{15}    % 15 * SWS
               %\UebAufw{9}{11}
	       %\PraAufw{2}{8}
	       %\Exam{50}\\
               %\Summe{120}

		\Praesenzzeit{36}    
                \VorNachbereitung{84}\\
                \Summe{120}
}

\Leistungsnachweis{
written exam, 2h duration, alternatively oral exam 
}

\Notenbildung{
Module grade is equal to grade of written exam
}

\Grundlagen{
Microfabrication lab (compulsory prerequisite)

Monolithic Microwave ICs in High-Speed Systems
(recommended)
}

% \Ilias{
% Optional.
% }

%\input{ausgabe}


%\end{document}
