\svnid{$Id$}

%\input{header}
%\begin{document}
%\input{macros}


%\input{header.tex}

\Modultitel{Biosensors}

\Modulkuerzel{70903}

\DeutscherTitel{-}

\SWS{
%An dieser Stelle werden die Präsenzzeiten angegeben (V/Ü)
%4+2 (V/Ü)
2
}

\ECTS{
%Anzahl der Leistungspunkte
3}

\Sprache{
%Unterrichtssprache
English 
}

\Moduldauer{1} %Hier steht ob das Modl ein oder zwei Semester lang ist.

\Turnus{
%\sporadisch{\WiSe 2010}
\periodisch{\WiSe} 
%\periodisch{\SoSe}
}

\Dozenten{
  %Ein Hochschullehrer, Honorarprofessor, Privatdozent, Gastprofessor oder ein
  %Lehrbeauftragter gemäß §44 LHG Baden-Württemberg. Hier stehen alle Hochschullehrer, 
  %die am Modul mitwirken.

   \LB{Dr. Alberto Pasquarelli}
  
}


\Modulverantwortlicher{
%Ein Dozent mit Prüfungsberechtigung, der ein Modul einrichtet und pflegt.
  \Prof{Dr.-Ing. Hermann Schumacher}
}

\Einordnung{
  \ET{\Ma}{\Wahlpflichtmodul}{\Ingwi}
  \ET{\Ma}{\Wahlmodul}{\AET}
  \ET{\Ma}{\Wahlmodul}{\KUS}
  \ET{\Ma}{\Wahlmodul}{\AUT}
  \ET{\Ma}{\Wahlmodul}{\Mikro}
  \Comm{\Ma}{\Wahlpflichtmodul}{\Mikro}
  }

\VoraussetzungenInhaltlich{
Basic knowledge of chemistry and biochemistry help understanding the biological part of biosensors.
}

\VoraussetzungenFormal{
keine
}

\Lernziele{
	%Hier soll ein Überbick über die fachlichen Inhalte gegeben werden.
Learning objectives:

The world-wide needs for chemical detection and analysis rises steadily. 
Several resons lead to this trend, for instance the rapid increase in the 
prevalence of diabetes, the increasing need for environmental and health 
monitoring, new legislative standards for food and drugs quality control 
or even the early detection of biological and chemical terror attacs. 
Thanks to higher sensitivity and specificity, short response times and 
reduction of overall costs, biosensors can be very competitive in 
addressing these needs when compared to traditional methods.  

Students can describe basic principles, mechanisms of action and 
applications of biosensors in different scenarios. After taking this 
module, participants can analyze biosensors, break-down in the elementary 
components and identify and illustrate every individual function in the 
information flow, from recognition to transduction and transmission. 
Students illustrate the clinical and industrial applications differentiate 
biosensor market sectors, e.g. commodities for everyday consumer needs or 
professional equipments for research. Furthermore, they are able to 
understand and critically analyze research in biosensors. Finally students 
are able to develop appropriate concepts and independently propose 
solutions for given problems.
}
	
\Inhalt{
	%Hier soll ein Überbick über die fachlichen Inhalte gegeben werden.
\spiegelstrich {Introduction to biosensors}
\spiegelstrich {Applications overview}
\spiegelstrich {Biological detection methods: catalithic, immunologic, etc}
\spiegelstrich {Physical transduction methods: electrochemical, optical, gravimetric, etc.}
\spiegelstrich {Immobilization techniques: adsorption, entrapment, cross-linking, covalent bonds}
\spiegelstrich {Biochip technologies: DNA and protein chips, Ion-channel devices, MEA and MTA, Implants}
\spiegelstrich {Extras: invited talk(s), experimental exercise, excursion}
}

\Literatur{ 

%Für das Modul empfohlene Bücher oder Aufsätze.
	\buch{Biosensors, Rai University } 
	\buch{Marks R.S. et al., Handbook of Biosensors and Biochips, Wiley, 2007} 
	\buch{Gizeli E. and Lowe C.R., Biomolecular Sensors, Taylor \& Francis, 2002} 
	\skript{Lecture Notes}
}

\Lehrformen{
\Vlg{``Biosensors'', lecture with demonstrations and seminars, 1,75~SWS (V)}{}
\Sem{``Biosensors'', 0,25~SWS}{}
\Lab{``Biosensors'', 2 x 2 h}{}
}

\Arbeitsaufwand{%Hier soll einheitlich die Präsenzzeit angegeben werden. Dabei wird kein
               %Unterschied zwischen Sommersemester und Wintersemester gemacht. 
               \Praesenzzeit{30}    % 15 * SWS
               \VorNachbereitung{50}
	       \Selbststudium{10}\\
               \Summe{90}
}

\Leistungsnachweis{
Written examination of 120 min.
}

\Notenbildung{
Module mark is identical to exam mark.
}

\Grundlagen{
Masters Thesis in the area of biosensors.
}

% \Ilias{
% Optional.
% }

%\input{ausgabe}


%\end{document}
