\svnid{$Id: $}

%\input{header}
%\begin{document}
%\input{macros}


%\input{header.tex}

\Modultitel{Using the Advanced Design System (ADS) in Electronic Design}

\Modulkuerzel{?????}

\DeutscherTitel{Einführung in das Advanced Design System (ADS)}

\SWS{
%An dieser Stelle werden die Präsenzzeiten angegeben (V/Ü)
%4+2 (V/Ü)
3
}

\ECTS{
%Anzahl der Leistungspunkte
4}

\Sprache{
%Unterrichtssprache
English 
}

\Moduldauer{1} %Hier steht ob das Modl ein oder zwei Semester lang ist.

\Turnus{
%\sporadisch{\WiSe 2010}
\periodisch{\WiSe} 
%\periodisch{\SoSe}
}

\Dozenten{
  %Ein Hochschullehrer, Honorarprofessor, Privatdozent, Gastprofessor oder ein
  %Lehrbeauftragter gemäß §44 LHG Baden-Württemberg. Hier stehen alle Hochschullehrer, 
  %die am Modul mitwirken.
  
  \LB{Dr.-Ing. Christoph Bromberger}
 
  
}


\Modulverantwortlicher{
%Ein Dozent mit Prüfungsberechtigung, der ein Modul einrichtet und pflegt.
  \LB{Dr.-Ing. Christoph Bromberger}
}

\Einordnung{
  \ET{\Ma}{\Wahlpflichtmodul}{\Ingwi}
  \IST{\Ma}{\Wahlpflichtmodul}{\Ingwi}
  \ET{\Ma}{\Wahlmodul}{\AET}
  \ET{\Ma}{\Wahlmodul}{\KUS}
  \Comm{\Ma}{\Wahlpflichtmodul}{\Mikro}
  }

\VoraussetzungenInhaltlich{
Bachelor-Level Analog Design Skills
Some understanding of S-parameters is helpful
}

\VoraussetzungenFormal{
keine
}

\Lernziele{
	%Hier soll ein Überbick über die fachlichen Inhalte gegeben werden.
The students gain an in-depth understanding of a significant analog simulation tool. 
They demonstrate their abilities to set up as well as to stream-line circuit 
simulations. Attendees employ ADS in high-frequency layout. They are used to ADS data 
structures and recognize ways to fully exploit their composition. Participants 
regularly scrutinize and critically judge their simulation results. 
}
	
\Inhalt{
	%Hier soll ein Überbick über die fachlichen Inhalte gegeben werden.
\spiegelstrich {The ADS project structure}
\spiegelstrich {Setting up, performing and simplifying schematics simulations}
\spiegelstrich {Using the data display}
\spiegelstrich {Understanding ADS data structures}
\spiegelstrich {Measurement Equations}
\spiegelstrich {Optimizing circuits with the help of ADS}
\spiegelstrich {(Semi-) automatic layout generation}
\spiegelstrich {2d electromagnetic simulation}
\spiegelstrich {Exporting the layout}
}

\Literatur{ 

%Für das Modul empfohlene Bücher oder Aufsätze.
	\buch{ADS handbooks and tutorials} 
	\aufsatz{http://edownload.soco.agilent.com/eedl/ads/2012\_08/zip/ADS2012PDF.zip} 
	\skript{A script is available for this lecture}
}

\Lehrformen{
\Vlg{``Using the Advanced Design System (ADS) in Electronic Design'', lecture , 2~SWS (V)}{}
\Ubg{``Using the Advanced Design System (ADS) in Electronic Design'', seminar, 1~SWS (V)}{} 
}

\Arbeitsaufwand{%Hier soll einheitlich die Präsenzzeit angegeben werden. Dabei wird kein
               %Unterschied zwischen Sommersemester und Wintersemester gemacht. 
               \Praesenzzeit{39}    % 15 * SWS
               \VorNachbereitung{56 }
	       \Selbststudium{25}\\
               \Summe{120}
}

\Leistungsnachweis{
Usually written exam of 120 minutes, otherwise oral exam; pre-requisite for participating 
in the exam: 60 per cent of the credits from the exercises
}

\Notenbildung{
Module mark is identical to exam mark.
}

\Grundlagen{
Masters Thesis in the area of biosensors.
}

% \Ilias{
% Optional.
% }

%\input{ausgabe}


%\end{document}
