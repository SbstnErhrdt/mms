\svnid{$Id$}

%\input{header}
%\begin{document}
%\input{macros}


%\input{header.tex}

\Modultitel{Lab - Vector Network Analysis}

\Modulkuerzel{71737}

\DeutscherTitel{-}


\SWS{
%An dieser Stelle werden die Präsenzzeiten angegeben (V/Ü)
%4+2 (V/Ü)
3
}

\ECTS{
%Anzahl der Leistungspunkte
5
}


\Sprache{
%Unterrichtssprache
Englisch
}

\Moduldauer{1} %Hier steht ob das Modl ein oder zwei Semester lang ist.

\Turnus{
%\sporadisch{\WiSe 2010}
\periodisch{\WiSe}
%\periodisch{\SoSe}
}

\Dozenten{
  %Ein Hochschullehrer, Honorarprofessor, Privatdozent, Gastprofessor oder ein
  %Lehrbeauftragter gemäß §44 LHG Baden-Württemberg. Hier stehen alle Hochschullehrer, 
  %die am Modul mitwirken.

  \Prof{Dr.-Ing. Hermann Schumacher}
  \LB{Dr.-Ing. Andreas Trasser}
  \LB{Dr.-Ing. Václav Valenta}
}


\Modulverantwortlicher{
%Ein Dozent mit Prüfungsberechtigung, der ein Modul einrichtet und pflegt.
  \Prof{Dr.-Ing. Hermann Schumacher}
}

\Einordnung{
  \ET{\Ma}{\Wahlpflichtmodul}{\Ingwi}
  \Comm{\Ma}{\Wahlpraktikum}{}
  \IST{\Ma}{\Wahlpraktikum}{}
}

\VoraussetzungenInhaltlich{
Basic knowledge of RF engineering

}

\VoraussetzungenFormal{
keine
 
}

\Lernziele{
%Hier soll ein Überbick über die fachlichen Inhalte gegeben werden. 
Students describe the most important concepts of vectorial measurements at radio 
frequencies and assess key tradeoffs between diverse measurement techniques, 
choice of  measurement parameters and error correction procedures. Students employ 
accurate calibration techniques and operate a vector network analyzer, 
distinguishing between frequency and time domain characterization techniques. 
They demonstrate time domain reflectometry using vector network analysis. 
Measurement results are interpreted and used to prepare equivalent circuit models 
of the measured components or to carry out deembeding of test fixtures of active 
devices.
}
		

\Inhalt{ 
	%Hier soll ein Überbick über die fachlichen Inhalte gegeben werden.
  Vector Network Analyzers (VNAs) are indispensable instruments in every RF
  laboratory as they provide the most common way to characterize network parameters 
  (e.g. scattering parameters, impedance or admittance parameters, etc.) of
  electrical networks (power amplifiers, filters and other n-port networks). 
  As a result, understanding the fundamental principles of VNA measurements 
  belongs to the essential knowledge of an RF engineer. The main goal of this 
  laboratory course is to introduce students the fundamental RF VNA measurement 
  techniques, principles, manipulations and measurement procedures. Throughout 
  different measurement exercises, this course will provide students firm grasp 
  and validation of the common theory gained in previous electro-technical
  courses.
  
List of experiments:  
 	\spiegelstrich{R, L, C measurements}
	\spiegelstrich{Measurements on a bias tee (RF/DC coupler)}
	\spiegelstrich{Measurements on a radio frequency filter}
	\spiegelstrich{Measurements on a semiconductor diode}
	\spiegelstrich{De-embedding procedures and 
 measurements of FET and BJT transistors in a test fixture
}
	\spiegelstrich{Measurements on coaxial cables (dielectric constant, length)}
	\spiegelstrich{Time domain reflectometry: measurements on passive components,
	localizations of faults in transmission linesTime domain reflectometry: measurements on passive }
	\spiegelstrich{Introduction to the time gating as a deembedding technique}
}	
	
	

\Literatur{ 
%Für das Modul empfohlene Bücher oder Aufsätze.
	Scattering parameter tutorial (Prof. Schumacher, provided online); 
	detailed descriptions for each experiment
	
	
    

}

\Lehrformen{

\Lab{``Vector Network Analysis'', 3~SWS}{}

}

\Arbeitsaufwand{%Hier soll einheitlich die Präsenzzeit angegeben werden. Dabei wird kein
               %Unterschied zwischen Sommersemester und Wintersemester gemacht. 
                %\VlgAufw{2}{0}    % 15 * SWS
                %\UebAufw{0}{36}
	        %\PraAufw{27}{85}\\
	        %\Summe{150}

		\Praesenzzeit{29}
		\VorNachbereitung{121}\\
		%\Selbststudium{}
		%\Gruppenarbeit{}
		\Summe{150}
}

\Leistungsnachweis{
} 

\Notenbildung{
The course is not graded. Successful completion requires
(a)	Sufficient preparation for each experiment (colloquium at start of each experiment)
(b)	Active participation in each experiment.
(c)	Submission and acceptance of report documenting each experiment (1 per group)
}

\Grundlagen{Master thesis with topics requiring vector network analysis

}
  
% \Ilias{
% Optional.
% }

%\input{ausgabe}


%\end{document}
