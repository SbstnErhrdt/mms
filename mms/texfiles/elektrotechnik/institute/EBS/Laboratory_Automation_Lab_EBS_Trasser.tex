\svnid{$Id$}

%\input{header}
%\begin{document}
%\input{macros}


%\input{header.tex}

\Modultitel{Lab - Laboratory Automation}

\Modulkuerzel{71731}

\DeutscherTitel{-}

\SWS{
%An dieser Stelle werden die Präsenzzeiten angegeben (V/Ü)
%4+2 (V/Ü)
4
}

\ECTS{
%Anzahl der Leistungspunkte
5
}


\Sprache{
%Unterrichtssprache
englisch
}

\Moduldauer{1} %Hier steht ob das Modl ein oder zwei Semester lang ist.

\Turnus{
%\sporadisch{\WiSe 2010}
\periodisch{\SoSe}
%\periodisch{\SoSe}
}

\Dozenten{
  %Ein Hochschullehrer, Honorarprofessor, Privatdozent, Gastprofessor oder ein
  %Lehrbeauftragter gemäß §44 LHG Baden-Württemberg. Hier stehen alle Hochschullehrer, 
  %die am Modul mitwirken.

  \LB{Dr.-Ing. Andreas Trasser}
  \LB{Dr. Václav Valenta}
}


\Modulverantwortlicher{
%Ein Dozent mit Prüfungsberechtigung, der ein Modul einrichtet und pflegt.
  \LB{Dr.-Ing. Andreas Trasser}
}

\Einordnung{
  \ET{\Ma}{\Wahlpflichtmodul}{\Ingwi}
  \Comm{\Ma}{\Wahlpraktikum}{\Mikro}\\
}


\VoraussetzungenInhaltlich{Fundamental knowledge of:
\spiegelstrich {Physics of semiconductor devices }
\spiegelstrich {RF engineering and twoport parameters}
\spiegelstrich {Programming (structuring problems and implementation in a programming language)}

}

\VoraussetzungenFormal{
%keine
}

\Lernziele{
%Hier soll ein Überbick über die fachlichen Inhalte gegeben werden. 
%Der Hörer ist fähig:
	
Students recognize and describe the electrical and logical principles of operation of the GPIB bus. 
The are able to develop simple programs to control measurement equipment, and to assist in the 
interpretation of measurement resuts. Students discover the functional principles of sampling 
oscilloscopes through experiments, and use it to conduct measurements on transmission lines. 
They identify the operational concepts of ve tor network analyzers, including error correction 
techniques, measure scattering parameters of microwave transistors, and discriminate the results 
from those obtained from an equivalent circuit. Students recognize fundamental noise measurement 
techniques, apply them to active devices in the frequency domain, and interpret them, interpreting 
differences between their measurements and theoretical models.
}
		

\Inhalt{
	%Hier soll ein Überbick über die fachlichen Inhalte gegeben werden.
	Students will get a practical view of special aspects of laboratory characterization techniques, with an application emphasis on 	the characterization of semiconductor devices. The experiments treat the DC characterization of semiconductor devices using source measure-units and GPIB control, noise parameter measurements, scattering parameter characterization, equivalent-time sampling oscillography, and time-domain reflectometry. \\

}
 
\Literatur{ 
%Für das Modul empfohlene Bücher oder Aufsätze.
	Script describing experiments including theoretical background 
}


\Lehrformen{
\Lab{``Laboratory Automation'', 4~SWS}{}  
}

\Arbeitsaufwand{%Hier soll einheitlich die Präsenzzeit angegeben werden. Dabei wird kein
               %Unterschied zwischen Sommersemester und Wintersemester gemacht. 
               %\VlgAufw{2}{0}    % 15 * SWS
               %\UebAufw{16}{32}
	    %\PraAufw{24}{124}\\
               %\Summe{150}
	    \Praesenzzeit{26}
	    \VorNachbereitung{124}\\
	    %\Selbststudium{}
	    %\Gruppenarbeit{}
	    \Summe{150}
}

\Leistungsnachweis{
Oral colloquium at start of each lab exercise (students may be excluded if insufficiently prepared; active participation in experiment; evaluation of written report for each experiment.
}

\Notenbildung{
The course is not graded. Successful completion requires
 \spiegelstrich {Sufficient preparation for each experiment (colloquium at start of each experiment)}
 \spiegelstrich {Active participation in each experiment}
 \spiegelstrich {Submission and acceptance of report documenting each experiment (1 per group)}

}

\Grundlagen{
Master thesis research involving electrical characterization techniques, especially related to semiconductor devices and active microwave components
}
  
% \Ilias{
% Optional.
% }

%\input{ausgabe}


%\end{document}
