\svnid{$Id$}

\Modultitel{Logik, Berechenbarkeit und Komplexität}

\Modulkuerzel{72016}

\EnglischerTitel{Logic, Computability and Complexity}

\SWS{6}

\ECTS{8}

\Sprache{Deutsch}

\Moduldauer{1}

\Turnus{
\periodisch{SoSe}
}

\Dozenten{
\Prof{Dr. Uwe Schöning}
\Prof{Dr. Jacobo Torán}
\Prof{Dr. Enno Ohlebusch}
}

\Modulverantwortlicher{\StudiendekanInf}

\Einordnung{
\Inf{\Ba}{\Pflichtfach}{\TMI}
\MedInf{\Ba}{\Schwerpunkt}{}
\SwEng{\Ba}{\Pflichtfach}{\SoftwareEngineering}
}

\VoraussetzungenInhaltlich{
Modul Formale Grundlagen, Modul Algorithmen und Datenstrukturen
}

\VoraussetzungenFormal{
Keine
}

\Lernziele{
Die Studierenden erwerben fundierte Kenntnisse zu den Grundlagen und der praktischen Relevanz der Logik unter besonderer Berücksichtigung der Informatik. Sie verstehen und erklären logisches Schlie\ss en. Die Studierenden können die vorgestellten Logikkalküle kritisch reflektieren, insbesondere hinsichtlich Komplexität, Korrektheit und Vollständigkeit. Sie sind in der Lage, Problemspezifikationen in Logikprogramme umzusetzen und beherrschen praktische Aspekte der Programmierung in Prolog. Die Studierenden erwerben fundierte Kenntnisse über die Formalisierung des Algorithmenkonzepts und des Effizienzbegriffs. Sie sind in der Lage, algorithmische Aufgabestellungen gemä\ss\ ihrer effizienten Lösbarkeit einzuordnen und besitzen anderseits Kenntnisse über die grundsätzlichen Grenzen der Algorithmisierbarkeit.
}

\Inhalt{
Im Modul werden Begriffe, Methoden und Resultate der formalen Logik vorgestellt, die in verschiedenen Gebieten der Informatik Anwendung finden. 
\spiegelstrich{Aussagenlogik: Syntax und Semantik, Normalformen, Erfüllbarkeitsproblem, Hornformeln und Markierungsalgorithmus, Resolution, Endlichkeitssatz.}
\spiegelstrich{Prädikatenlogik erster Stufe: Syntax und Semantik, Normalformen, Skolemform, Erfüllbarkeitsproblem, Formalisierung des Folgerungsbegriffs, Unifikation, Korrektheit und Vollständigkeit des Resolutionskalküls.}
\spiegelstrich{Logische Programmierung: Theoretische Grundlagen und praktische Programmierung in Prolog.}Im Modul werden die grundlegenden Begriffe aus den Gebieten der Berechenbarkeits- und Komplexitätstheorie eingeführt. 
\spiegelstrich{Aussagenlogik: Syntax und Semantik, Normalformen, Erfüllbarkeitsproblem, Hornformeln und Markierungsalgorithmus, Resolution, Endlichkeitssatz.}
\spiegelstrich{Prädikatenlogik erster Stufe: Syntax und Semantik, Normalformen, Skolemform, Erfüllbarkeitsproblem, Formalisierung des Folgerungsbegriffs, Unifikation, Korrektheit und Vollständigkeit des Resolutionskalküls.}
\spiegelstrich{Logische Programmierung: Theoretische Grundlagen und praktische Programmierung in Prolog.}
}

\Literatur{
\spiegelstrich{Skript: Vorlesungsskript Logik}
\spiegelstrich{Buch: U. Schöning: Logik für Informatiker. 5. Auflage, Spektrum Verlag, 2000}
\spiegelstrich{Buch: J. Dassow: Logik für Informatiker. Teubner, 2005}
\spiegelstrich{Buch: B. Heinemann, K. Weihrauch: Logik für Informatiker. Teubner, 1992.}
\spiegelstrich{Skript: Vorlesungsskript berechenbarkeit und Komplexität}
\spiegelstrich{Buch: U. Schöning: Theoretische Informatik -- kurz gefasst, Spektrum Verlag, 5. Auflage, 2008.}
\spiegelstrich{Buch: M. Sipser: Introduction to the Theory of Computation; PWS Publ. Company, 1997.}
}

\Lehrformen{
\Vlg{Logik}{\Prof{}}
\Ubg{Logik}{\Prof{}}
\Vlg{Berechenbarkeit und Komplexität}{\Prof{}}
\Ubg{Berechenbarkeit und Komplexität}{\Prof{}}
}

\Arbeitsaufwand{
\Praesenzzeit{90}
\VorNachbereitung{150}
\Summe{240}
}

\Leistungsnachweis{
Die Modulprüfung erfolgt schriftlich.
}

\Notenbildung{
Die Modulnote ergibt sich aus der Modulprüfung. Bei erfolgreicher Teilnahme an den Übungen wird dem Studierenden ein Notenbonus gemäß §13 (5) der fachspezifischen Prüfungsordnung Informatik/Medieninformatik/Software Engineering gewährt.
}

\Grundlagen{
--
}