\svnid{$Id$}

\Modultitel{Anleitung zum Erstellen von Modulen}

\Modulkuerzel{AEM / }

\EnglischerTitel{How to write modules}

\SWS{
%An dieser Stelle werden die Präsenzzeiten angegeben (V/Ü)
4+4 (V/Ü)
}

\ECTS{
%Anzahl der Leistungspunkte
6}

\Sprache{
%Unterrichtssprache
Deutsch
}

\Moduldauer{1} %Hier steht ob das Modl ein oder zwei Semester lang ist.

\Turnus{
%\sporadisch{\WiSe 2010}
\periodisch{\WiSe}
%\periodisch{\SoSe}
}

\Dozenten{
  %Ein Hochschullehrer, Honorarprofessor, Privatdozent, Gastprofessor oder ein
  %Lehrbeauftragter gemäß §44 LHG Baden-Württemberg. Hier stehen alle Hochschullehrer,
  %die am Modul mitwirken.

  \Prof{Dr.-Ing. Frank Slomka}
  \PD{Dr. Hans Armin Kestler}
  \LB{Dr. Mohamed Oubbati}
}


\Modulverantwortlicher{
%Ein Dozent mit Prüfungsberechtigung, der ein Modul einrichtet und pflegt.
  \Prof {Dr.-Ing. Frank Slomka}
}

\Einordnung{
  \Inf{\Ba}{\Pflichtfach}{Praktische Informatik}
  \Inf{\Ba}{\Pflichtfach}{Technische Informatik}
  \Inf{\Ba}{\Pflichtfach}{Formale Grundlagen der Informatik}
  \Inf{\Ba}{\Schwerpunkt}{Eingebettete Systeme}
  \Inf{\Ba}{\Pflichtfach}{Mathematik}
  \MedInf{\Ba}{\Proseminar}{Eingebettete Systeme}
  \Inf{\Ba}{\Seminar}{Eingebettete Systeme}
  \Inf{\Ba}{\Anwendungsfach}{Physik}
  \Inf{\Ma}{\ASQ}{Labern für Anfänger}
  \Inf{\Ma}{\Kernfach}{Praktische Informatik}
  \Inf{\Ma}{\Pflichtfach}{Technische Informatik}
  \Inf{\Ma}{\Pflichtfach}{Formale Grundlagen der Informatik}
  \MedInf{\Ma}{\Vertiefungsfach}{Eingebettete Systeme}
  \SwEng{\Ba}{\Schwerpunkt}{Eingebettete Systeme}
  \IST{\Ma}{\Pflichtmodul}{Eingebettete Systeme}
  \ET{\Ma}{\Wahlmodul}{Eingebettete Systeme}
  \Comm{\Ma}{\Wahlpflichtmodul}{Embedded Systems}
}

\VoraussetzungenInhaltlich{
Hier stehen Informationen, welche Inhalte den Studierenden bekannt sein
sollten.
}

\VoraussetzungenFormal{
Hier können formale Voraussetzungen spezifiziert werden.
}

\Lernziele{
Hier soll spezifiziert werden, was ein Studierender nach dem Besuch
des Moduls und Ablegen der Prüfung kann
}

\Inhalt{
	%Hier soll ein Überbick über die fachlichen Inhalte gegeben werden.
	\spiegelstrich {Thema 1}
	\spiegelstrich {Thema 2}
}

\Literatur{

%Für das Modul empfohlene Bücher oder Aufsätze.}
	\buch{Albert Einstein: Allgemeine Relativitätstheorie, Vieweg 1936}
	\aufsatz{Albert Einstein: Zur Elektrodynamik bewegter Körper, Annalen der
	Physik 1905}
	\skript{Vorlesungsskript}
}

\Lehrformen{
\Vlg{Technische Informatik}{Prof. Dr. Heiko Falk}
\Ubg{Entwurfsmethodik Eingebetteter Systeme}{Dipl.-Ing. Tobias Bund}
\Tut{Entwurfsmethodik Eingebetteter Systeme}{Diverse}
\Lab{Entwurfsmethodik Eingebetteter Systeme}{Dipl.-Ing. Tobias Bund}
\Prj{Echtzeitkommunikationssysteme}{Dipl.-Inf. Steffen Moser}
\Sem{Eingebettete Systeme}{Prof. Dr.-Ing. Frank Slomka}
\ProSem{Eingebettete Systeme}{Dipl.-Inf. Victor Pollex}
}

\Arbeitsaufwand{%Hier soll einheitlich die Präsenzzeit angegeben werden. Dabei wird kein
               %Unterschied zwischen Sommersemester und Wintersemester gemacht.
               \Praesenzzeit{60}    % 15 * SWS
               \VorNachbereitung{120}
               \Summe{180}
}

\Leistungsnachweis{
Hier werden Angaben gemacht bzgl. des Erwerbs der Leistungspunkte. Dafür
können Vorleistungen und Prüfungsformen definiert werden.
}

\Notenbildung{
Hier wird beschrieben, wie die Note ermittelt wird.
}

\Grundlagen{
Optional.
}

\Ilias{
Optional.
}

