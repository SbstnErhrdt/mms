\svnid{$Id$}

%\input{header.tex}

\Modultitel{Einführung in die Informatik}

\Modulkuerzel{70319}

\EnglischerTitel{Introduction to Computer Science}

\SWS{
%An dieser Stelle werden die Präsenzzeiten angegeben (V/Ü)
%4+2 (V/Ü)
6
}

\ECTS{
%Anzahl der Leistungspunkte
8}

\Sprache{
%Unterrichtssprache
Deutsch
}

\Moduldauer{1} %Hier steht ob das Modl ein oder zwei Semester lang ist.

\Turnus{
%\sporadisch{\WiSe 2010}
\periodisch{Semester}
%\periodisch{\SoSe}
}

\Dozenten{
  %Ein Hochschullehrer, Honorarprofessor, Privatdozent, Gastprofessor oder ein
  %Lehrbeauftragter gemäß §44 LHG Baden-Württemberg. Hier stehen alle Hochschullehrer, 
  %die am Modul mitwirken.

  \Prof{Dr. Susanne Biundo}
  \Prof{Dr. Peter Dadam}
  \Prof{Dr. Enno Ohlebusch}
  \Prof{Dr. Heiko Neumann}
  \Prof{Dr. Manfred Reichert}
  \JunProf{Dr. Birte Glimm}
}


\Modulverantwortlicher{
%Ein Dozent mit Prüfungsberechtigung, der ein Modul einrichtet und pflegt.
  \StudiendekanInf
}

\Einordnung{
  \Inf{\Ba}{\Pflichtfach}{\PAI}
  \MedInf{\Ba}{\Pflichtfach}{\PAI}
  \SwEng{\Ba}{\Pflichtfach}{\PAI}
  \IST{\Ba}{\Pflichtmodul}{}
  \Inf{\La}{\Pflichtfach}{}
}

\VoraussetzungenInhaltlich{
Keine
}

\VoraussetzungenFormal{
Keine
}

\Lernziele{
Die Studierenden sind in der Lage, elementare Konzepte und Methoden der Informatik zu beschreiben. 
Sie können eine erste Programmiersprache beurteilen und durch deren praktischen Gebrauch überschaubare Problemstellungen lösen. 
Die Studierenden können grundlegende Datenstrukturen (Arrays, Listen, Bäume, Graphen), elementare Strukturierungs- und Verarbeitungsmechanismen (Objektorientierung, Modularisierung, Divide-and-Conquer, Iteration, Rekursion) sowie Standardalgorithmen zum Suchen und Sortieren benennen und beschreiben. 
Die Studierenden können formale Beschreibungsmittel interpretieren und sind in der Lage diese zu bewerten. Sie können ferner Programme mit Hilfe elementarer Komplexitätsanalysen analysieren und beurteilen.
}

\Inhalt{
	%Hier soll ein Überbick über die fachlichen Inhalte gegeben werden.
	\spiegelstrich {Elementare Konzepte, Prinzipien und Methoden der Informatik}
	\spiegelstrich {Grundkenntnisse im Programmieren einer objektorientierten Sprache am Beispiel von Java} 
	\spiegelstrich {Definition des Begriffs Algorithmus}
	\spiegelstrich {Grundprinzipien des Software Engineering}
	\spiegelstrich {Grundkonzepte imperativer Programmiersprachen (Syntax, Semantik, elementare Datentypen, Daten- und Kontrollstrukturen)}
	\spiegelstrich {Grammatikformalismen}
	\spiegelstrich {Dynamische Datenstrukturen und ihre Verarbeitung (Listen, Bäume, Graphen, Rekursion)}
	\spiegelstrich {Konzepte der Objektorientierung (Kapselung, Vererbung)}
	\spiegelstrich {Elementare Such- und Sortieralgorithmen}
	\spiegelstrich {Komplexität (Effizienz von Algorithmen, O-Notation)}
% 	\spiegelstrich Korrektheit von Programmen (Hoare-Kalkül).
}

\Literatur{ 

%Für das Modul empfohlene Bücher oder Aufsätze.
	\skript{Vorlesungsskript}
	\buch{Gumm Heinz-Peter, Sommer Manfred: Einführung in die Informatik, Oldenbourg Verlag, 2006} 
	\buch{Broy Manfred: Informatik - Eine grundlegende Einführung, Band 1, Programmierung und Rechnerstrukturen, Springer Verlag, 1998} 
	\buch{Küchlin Wolfgang, Weber Andreas: Einführung in die Informatik - Objektorientiert mit Java, Springer Verlag, 2003} 
	\buch{Echtle Klaus, Goedicke Michael: Lehrbuch der Programmierung mit Java, dpunkt Verlag, 2000}
}

\Lehrformen{
\Vlg{Praktische Informatik}{4~SWS (V), 5~ECTS}
\Ubg{Praktische Informatik}{2~SWS (Ü), 3~ECTS}
}

\Arbeitsaufwand{%Hier soll einheitlich die Präsenzzeit angegeben werden. Dabei wird kein
               %Unterschied zwischen Sommersemester und Wintersemester gemacht. 
               \Praesenzzeit{90}    % 15 * SWS
               \VorNachbereitung{150}
               \Summe{240}
}

\Leistungsnachweis{
Die Modulprüfung erfolgt in Form einer schriftlichen Klausur.

Die Vergabe der Leistungspunkte erfolgt aufgrund der erfolgreichen Teilnahme an den Übungen (3~ECTS) und des Bestehens einer schriftlichen Prüfung zur Vorlesung (5~ECTS). 
}

\Notenbildung{
Die Modulnote ergibt sich aus der Modulprüfung.
}

\Grundlagen{
Das Modul bildet die Grundlage für die Module Programmieren von Systemen, Algorithmen und Datenstrukturen, Paradigmen der Programmierung. Wünschenswert ist es dieses Modul vor dem Besuch eines Seminars abgeschlossen zu haben.
}

% \Ilias{
% Optional.
% }

