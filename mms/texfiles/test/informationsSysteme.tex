\svnid{$Id$}

%\input{header.tex}

\Modultitel{Informationssysteme}

\Modulkuerzel{71430}

\EnglischerTitel{Information Systems}

\SWS{
%An dieser Stelle werden die Präsenzzeiten angegeben (V/Ü)
%2+2 (V/Ü)
4
}

\ECTS{
%Anzahl der Leistungspunkte
6}

\Sprache{
%Unterrichtssprache
Deutsch
}

\Moduldauer{1} %Hier steht ob das Modl ein oder zwei Semester lang ist.

\Turnus{
%\sporadisch{\WiSe 2010}
\periodisch{\SoSe}
%\periodisch{\SoSe}
}

\Dozenten{
  %Ein Hochschullehrer, Honorarprofessor, Privatdozent, Gastprofessor oder ein
  %Lehrbeauftragter gemäß §44 LHG Baden-Württemberg. Hier stehen alle Hochschullehrer, 
  %die am Modul mitwirken.
  
  \Prof{Dr. Peter Dadam}
  \Prof{Dr. Manfred Reichert}
}


\Modulverantwortlicher{
%Ein Dozent mit Prüfungsberechtigung, der ein Modul einrichtet und pflegt.
\StudiendekanInf
}

\Einordnung{
  \Inf{\Ba}{\Pflichtfach}{\PAI}
  \MedInf{\Ba}{\Pflichtfach}{\PAI}
  \SwEng{\Ba}{\Pflichtfach}{\PAI}
  \IST{\Ba}{\Wahlpflichtmodul}{}
  \Inf{\La}{\Pflichtmodul}{}
  \ET{\Ba}{\Nebenfach}{Informatik}
}

\VoraussetzungenInhaltlich{Modul Einführung in die Informatik, Modul Programmieren von Systemen und Modul Paradigmen der Programmierung
}

\VoraussetzungenFormal{
keine
}

\Lernziele{
Die Studierenden können die Grundlagen verschiedener Basistechnologien zur Implementierung von (betrieblichen) Informationssystemen beschreiben und beurteilen. Sie können darüber hinaus erklären, wie auf dieser Grundlage konventionelle und prozessorientierte Informationssysteme realisiert werden.
}

\Inhalt{
	%Hier soll ein Überbick über die fachlichen Inhalte gegeben werden.
	\spiegelstrich {Vertiefung relationaler Datenbanken}
	\spiegelstrich {Entwicklung datenbankbasierter Informationssysteme mit
	relationalen Datenbanksystemen} 
	\spiegelstrich {Realisierung prozessorientierter Informationssysteme und
	Prozess-Management-Technlogien} 
	\spiegelstrich {Dokumenten-Management-Systeme und ihre Anwendung}
	\spiegelstrich {XML-Unterstützung in Datenbanksystemen}
	\spiegelstrich {Prozessorientierte Systemintegration}
}

\Literatur{ 

%Für das Modul empfohlene Bücher oder Aufsätze.}
	\skript{Vorlesungsskript}
	%\item[] \hspace{-1em}Themenbereich Datenbank-Management-Systeme:
	\buch{A. Kemper, A. Eickler: Datenbanksysteme - eine Einführung, 7. Aufl., Oldenbourg, 2009} 
	\buch{A. Kemper, M. Wimmer: Übungsbuch Datenbanksysteme, 2. Aufl., Oldenbourg, 2009} 
	\buch{Elmasri, S. Navathe: Grundlagen von Datenbanksystemen, Pearson Studium, 2005}
	%\item[] \hspace{-1em}Themenbereich Prozess-Management-Systeme:
	\buch{B. Baumgarten: Petri-Netze - Grundlagen und Anwendungen. 2. Auflage. Spektrum Akademischer Verlag, 1996}
	\buch{J. Becker, C. Mathas, A. Winkelmann: Geschäftsprozessmanagement, Springer, 2009}
	\buch{J. Staudt: Geschäftsprozessanalyse, Springer, 3. Auflage, 2006}
	\buch{M. Weske: Business Process Management: Concepts, Languages, Architectures, 2007}
	%\item[] \hspace{-1em}Themenbereich Dokumenten-Management-Systeme: 
	\buch{J. Gulbins, M. Seyfried, H. Strack-Zimmermann: Dokumenten-Management: Vom Imaging zum Business-Dokument, 3. Aufl., Springer, 2002}
	\buch{K. Götzer, R. Schmale, B. Maier, T. Komke: Dokumenten-Management: Informationen im Unternehmen effizient nutzen, 4. Aufl., dpunkt-Verlag, 2008}
}

\Lehrformen{
\Vlg{Informationssysteme, 2~SWS}{}
\Ubg{Informationssysteme, 2~SWS}{}
% \Lab{Entwurfsmethodik Eingebetteter Systeme}{Dipl.-Ing. Tobias Bund}
% \Prj{Echtzeitkommunikationssysteme}{Dipl.-Inf. Steffen Moser}
% \Sem{Eingebettete Systeme}{Prof. Dr.-Ing. Frank Slomka}
% \ProSem{Eingebettete Systeme}{Dipl.-Inf. Victor Pollex}
}

\Arbeitsaufwand{%Hier soll einheitlich die Präsenzzeit angegeben werden. Dabei wird kein
               %Unterschied zwischen Sommersemester und Wintersemester gemacht. 
               \Praesenzzeit{60}    % 15 * SWS
               \VorNachbereitung{120}
               \Summe{180}
}

\Leistungsnachweis{
Die Modulprüfung erfolgt in Form einer schriftlichen Klausur.
}

\Notenbildung{Die Modulnote ergibt sich aus der Modulprüfung.
Bei erfolgreicher Teilnahme an den Übungen wird dem Studierenden ein Notenbonus gemäß \S13 (5) der fachspezifischen Prüfungsordnung Informatik/Medieninformatik/Software Engineering gewährt.
}

\Grundlagen{
Weiterführende Veranstaltungen in des jeweiligen Bachelor-Studiengangs. }
