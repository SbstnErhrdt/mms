\svnid{$Id$}

\Modultitel{Logik}

\Modulkuerzel{71218}

\EnglischerTitel{Logic}

\SWS{3}

\ECTS{4} 

\Sprache{Deutsch}

\Modulverantwortlicher{\StudiendekanInf}

\Moduldauer{1}

\Turnus{
\periodisch{\SoSe}
}

\Dozenten{
\Prof{Dr. Uwe Sch\"oning}
\Prof{Dr. Jacobo Tor\'an}
\Prof{Dr. Enno Ohlebusch}
}
\Einordnung{
\Inf{\La}{\Pflichtmodul}
}

\VoraussetzungenInhaltlich{Modul Formale Grundlagen}
\VoraussetzungenFormal{Keine}

\Lernziele{Die Studierenden erwerben fundierte Kenntnisse zu den Grundlagen 
und der praktischen Relevanz der Logik unter besonderer Ber\"ucksichtigung der 
Informatik. Sie verstehen und erkl\"aren logisches Schlie\ss en.
Die Studierenden k\"onnen die vorgestellten Logikkalk\"ule kritisch reflektieren,
insbesondere hinsichtlich Komplexit\"at, Korrektheit und 
Vollst\"andigkeit. Sie sind in der Lage, Problemspezifikationen in
Logikprogramme umzusetzen und beherrschen praktische Aspekte der
Programmierung in Prolog.
}

\Inhalt{
\spiegelstrich{Im Modul werden Begriffe, Methoden und Resultate der formalen 
Logik vorgestellt, die in verschiedenen Gebieten der Informatik Anwendung 
finden.}
\spiegelstrich{Aussagenlogik: Syntax und Semantik, Normalformen,
Erf\"ullbarkeitsproblem, Hornformeln und Markierungsalgorithmus, Resolution, 
Endlichkeitssatz.}
\spiegelstrich{Pr\"adikatenlogik erster Stufe: Syntax und Semantik,
Normalformen, Skolemform, Erf\"ullbarkeitsproblem, Formalisierung des 
Folgerungsbegriffs, Unifikation, Korrektheit und Vollst\"andigkeit des 
Resolutionskalk\"uls.}
\spiegelstrich{Logische Programmierung: Theoretische Grundlagen und praktische
Programmierung in Prolog.}
}

\Literatur{
\skript{Vorlesungsskript}
\buch{U. Sch\"oning: Logik f\"ur Informatiker. 5. Auflage, Spektrum Verlag, 2000}
\buch{J. Dassow: Logik f\"ur Informatiker. Teubner, 2005}
\buch{B. Heinemann, K. Weihrauch: Logik f\"ur Informatiker. Teubner, 1992.}
}

\Grundlagen{--}

\Lehrformen{
\Vlg{Logik, 2 SWS}{}
\Ubg{Logik, 1 SWS}{}
}

\Arbeitsaufwand{
\Praesenzzeit{45}    
\VorNachbereitung{75}
\Summe{120}
}

\Leistungsnachweis{Die Modulpr\"ufung erfolgt schriftlich.}

\Notenbildung{Die Modulnote ergibt sich aus der Modulprüfung.
Bei erfolgreicher Teilnahme an den Übungen wird dem Studierenden ein Notenbonus gemäß %$17 Abs. 3a der Rahmenordnung gewährt.
}

